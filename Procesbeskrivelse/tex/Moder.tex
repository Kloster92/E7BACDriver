\chapter{Møder}
Der har været afholdt tre forskellige slags møder i løbet af processen. Det indebærer interne gruppemøder, vejledermøde med Arne Justesen og eksterne møder med kontaktpersonerne fra Terma, Johnny Laursen og Hans Jensen.  

De interne møder har været brugt hver dag i form af det daglige scrum møde, også kaldet "stå op møde". Her begyndes dagen med at hvert medlem reflektere over gårsdagens arbejde, og hvad der skal indtil næste møde. Eventuelle problemer med en given opgave opgives her, og findes der ikke en hurtig løsning, gås der i dybden med det efter mødet. 

Langt hen af vejen har der været afholdt vejledermøde en gang i ugen. Her har der forinden været sendt en dagsorden til vejleder, der beskriver hvad mødet vil handle om, og måske indeholde nogle dokumenter der skal gennemlæses inden mødet. 

Møderne har indeholdt en gennemgang af fremskridt den seneste uge, og planen for den næste uge. Derudover har der været uddybende snakke omkring tekniske og rapportmæssige problemer, der opstod undervejs. Der har ikke været deciderede sprint retrospectives, da der har været afholdt møder i hver uge, og sprintsne har varet mere end det. De ugentlige møder minder dog meget om det, da der har været snak om den forgangne uge ved hvert møde og udbyttet af det, ligesom ved retrospectives. 

Til hvert møde har der været brugt en referent, der har haft til opgave at fange og skrive essensen af samtalerne. Derudover har det andet gruppemedlem fungeret som mødeleder, og har haft til ansvar at lede mødet med henblik på dagsordenen. På den måde sikres det både, at mødet gennemgår alle de tænkte punkter og det hele bliver dokumenteret. Mødeleder og referent er gået på skift fra møde til møde.    

I en måned midtvejs i projektet har projektets vejleder været sygemeldt, og Emir Pasic har været standin i den periode. Da hans kendskab til selve stoffet er begrænset, var det mere den rapportmæssige del han kunne bidrage til. Da projektet i denne periode var i design-, implementerings- og testfasen, blev det meste tid brugt på Terma, og derfor blev udbyttet af standin-vejlederen begrænset.      

Der har været afholdt møder med kontaktpersonerne fra Terma næsten hver uge. Igen har der været sendt information til kontaktpersonerne om, hvad der ønskes at opnå til møderne. Til disse møder har der været rigtig god mulighed for, at diskutere de faglige problemer, der er opstået undervejs. Igen har der været brugt en mødeleder og en referent, som også her er gået på skift.  
I design-, implementerings- og testfasen har der dagligt været faglige diskussioner. Det har været produktivt, at kunne få svar på et problem med det samme, i stedet for at skulle vente på det næste ugentlige møde. Mødereferater kan findes i den vedlagte bilagsmappe. 