\chapter{Indledning}
Proces er en særdeles vigtig del af et projektforløb, og består af mange forskellige elementer. En god processtruktur kan hjælpe til at give bedre overblik over perioden og gøre alle parter enige om, hvordan rapport og produkt skal udformes. Uden elementerne i en procesbeskrivelse risikerer gruppen, at arbejde i forskellige retninger, og på den måde ikke opnå optimalt samarbejde. Ved at have en række retningslinjer skrevet ned omkring, hvordan f.eks. arbejdsfordelingen og udviklingsforløbet skal foregå, er risikoen for misforståelser mindre i løbet af perioden. En veldefineret processtruktur bliver vigtigere og vigtigere, jo flere folk der arbejder sammen. Men selv i grupper af få personer, giver det optimerede arbejdsbetingelser, når der er faste måder, at gøre tingene på.  

I denne gruppe er der lagt vægt på, at opsætte en proces, der giver det bedste fundament for en god produktudvikling. Da gruppen består af to medlemmer har der med fastlagte udviklingsforløb og arbejdsfordelinger, været mere tid til udviklingen af produktet. Der er gennemgående benyttet en iterativ udviklingsproces, da det har været nødvendigt at opnå erfaring og ny viden løbende. 

Grundet gruppestørrelsen har der ikke været fordelt procesmæssige hovedansvarsområder blandt medlemmerne. Der har istedet været en flad struktur, hvor begge medlemmer har været inde over alle delene i processen.   