\chapter{Arbejdsfordeling}
Arbejdsfordelingen i forbindelse med produktets udvikling har været en blanding imellem, at kunne anvende forhåndsviden, og udfordre os selv ved at arbejde i ukendte områder. Der er startet op med en solid baggrundsviden fra faget "Effektelektronik", som gav kendskab til forskellige converter typer, så der forinden var et teoretisk grundlag at arbejde ud fra. Til gengæld har det vist sig, at udarbejdning af en converter i praksis bød på mange ting, som på forhånd var ukendt stof for gruppen.  

Det har givet en god mellemvej mellem kendte og ukendte ting, hvor en masse nyt stof er erfaret undervejs, og hvor gruppen ikke alt for ofte har befundet sig på bar bund, men har kunnet drage paralleller til undervisningen.

Til selve fordelingen af arbejdsopgaver, er der brugt Scrum taskboard, som nævnt i sektion~\ref{udvikling}. Det har fungeret ganske godt, da det selv med en gruppe på to, er vigtigt at kunne holde et godt overblik. Samtidig kan man altid sikre sig, at der ikke er to i gang med den samme opgave, ved at kigge på taskboardet. Det er også en god måde, at holde øje med at alle medlemmer tager nok opgaver, og at det ikke kun er en enkelts navn, der står på dem alle. Igen har det ikke været et problem i dette projekt, hvor opgave fordelingen har været helt lige.     