\chapter{Gruppedannelse}
Gruppen er sammensat ud fra samarbejde igennem de tidligere semestre. Vi har gennemgående haft samme kurser, og har hele vejen igennem haft et godt samarbejde. Det var derfor naturligt, at skrive bacheloren sammen. Det giver samtidig også en fordel, at der på forhånd er kendskab til hinandens styrker og svagheder.


Gruppen består af 2 medlemmer, hvilket er et bevist valg, da der på den måde er større mulighed for at forme projektet og processen efter individuelle ønsker. Der er både fordele og ulemper ved at have flere inputs. Her har det været en succes at være få, da der ikke er brugt unødig tid på at diskutere småting, men istedet have tid og overskud til, at diskutere de vigtige valg mere i dybden. Ved de tidligere 4 semesterprojekter, hvor grupperne har været omkring 8 personer, har der været meget forskellige meninger om, hvordan tingene skal gøres, hvilket sløver processen. Da vi har været i gruppe sammen i 3 af tilfældene, har det været nemt at vurdere, hvad der tidligere har fungeret. Det gør, at der hurtigt nås til enighed om den optimale måde at gøre tingene på.
