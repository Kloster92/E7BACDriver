\chapter{Samarbejdskontrakt}
Der er udarbejdet en samarbejdskontrakt, som kan findes i bilagsmappen. Samarbejdskontrakten definerer dele af strukturen for gruppearbejdet såsom møder. Kontrakten er mest af alt en formalitet, da antallet af gruppemedlemmer er på 2, og derfor væsentlig færre end ved tidligere semesterprojekter. Ved større grupper er der i højere grad brug for skriftlige aftaler, da der ellers kan være medlemmer som gemmer sig, eller bliver udeladt i processen. Det er ikke det store problem i en gruppe af 2 personer. Til gengæld giver det stadig en sikkerhed og en bekræftelse af, at man har de samme forventninger til udviklingsprocessen og arbejdsmentaliteten. Det giver også mulighed for at opstille rammer for løsninger af eventuelle konflikter, så disse løses bedst muligt. 

I dette projekt er samarbejdsaftalen blevet overholdt og det har ikke været nødvendigt at finde den frem på noget tidspunkt.  