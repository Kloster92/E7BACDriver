\chapter{Accepttest}


%% Skabelon
%\begin{table}[H] 			
%	\centering
%	\begin{tabularx}{\textwidth}{|c|X|X|X|X|}
%		\hline
%		\bfseries Use case under test & \multicolumn{4}{|c|}{< Use case >} \\ \hline
%		\bfseries Scenarie & \multicolumn{4}{| c |}{< Scenarie >} \\ \hline
%		\bfseries Prækondition &  \multicolumn{4}{|c|}{< Prækondition >} \\  \hline
%		\bfseries Step  & \bfseries Handling &  \bfseries Forventet & \bfseries Faktisk & \bfseries Vurdering \\ \hline 
%		\end{tabularx}
%		\caption{< Caption >}
%\end{table}

\section{Tests}
\begin{table}[H] 			
	\centering
	\begin{tabularx}{\textwidth}{|c|X|X|X|X|}
		\hline
		\bfseries Use case under test & \multicolumn{4}{|c|}{Use case 1 - Aktiver Thermal Knife load} \\ \hline
		\bfseries Scenarie & \multicolumn{4}{| c |}{Hovedscenarie} \\ \hline
		\bfseries Prækondition &  \multicolumn{4}{|c|}{Hverken Use case 1 eller Use case 2 er under udførelse} \\  \hline
		\bfseries Step  & \bfseries Handling &  \bfseries Forventet & \bfseries Faktisk & \bfseries Vurdering \\ \hline 
		1 & Brugeren vælger Thermal Knife load & Reb bliver brændt over & & \\ \hline
	\end{tabularx}
	\caption{Test for Use case 1 - Start bil - Hovedscenarie}
\end{table}

\begin{table}[H] 			
	\centering
	\begin{tabularx}{\textwidth}{|c|X|X|X|X|}
		\hline
		\bfseries Use case under test & \multicolumn{4}{|c|}{Use case 2 - Aktiver Pyro load} \\ \hline
		\bfseries Scenarie & \multicolumn{4}{| c |}{Hovedscenarie} \\ \hline
		\bfseries Prækondition &  \multicolumn{4}{|c|}{Hverken Use case 1 eller Use case 2 er under udførelse} \\  \hline
		\bfseries Step  & \bfseries Handling &  \bfseries Forventet & \bfseries Faktisk & \bfseries Vurdering \\ \hline 
		1 & Brugeren vælger Pyro load & Krudtladning bliver antændt & & \\ \hline
	\end{tabularx}
	\caption{Test for Use case 1 - Start bil - Hovedscenarie}
\end{table}


\subsection{Test af ikke-funktionelle krav}

\begin{tabularx}{\textwidth}{|X|X|X|X|X|}
	\hline
	\textbf{Krav} & \textbf{Test} & \textbf{Forventet resultat} & \textbf{Resultat} & \textbf{Vurdering} \\ \hline
	Input-spændingen skal være mellem 26-100V & Indgangs-spændingen måles med et voltmeter & Indgangs-spændingen er mellem 26-100V && \\ \hline
	Der må maksimalt trækkes en peak-strøm fra inputkilden på 150\% af inputstrømmen & Udgangen belastes af en 3\ohm\ modstand, og der måles strøm på indgangen med oscilloskop & Peakstrømmen overstiger ikke 150\% af steady state strømmen & & \\ \hline
	Skal opretholde en outputspænding på op til 21V $\pm$ 2\% ved 2,5A $\pm$ 5\% & Der indsættes en load på 5\ohm\ og udgangs-strøm og -spænding måles med oscilloskop & Spændingen ligger på 12,5V $\pm$ 2\% og strømmen på 2,5A $\pm$ 5\% && \\ \hline
	Skal opretholde en outputstrøm op til 5A $\pm$ 5\% ved 15V $\pm$ 2\% & Der indsættes en load på 5\ohm\ og udgangs-strøm og -spænding måles med oscilloskop & Spændingen ligger på 15V $\pm$ 2\% og strømmen på 3A $\pm$ 5\% && \\ \hline
\end{tabularx}



\begin{tabularx}{\textwidth}{|X|X|X|X|X|}
	\hline
	\textbf{Krav} & \textbf{Test} & \textbf{Forventet resultat} & \textbf{Resultat} & \textbf{Vurdering} \\ \hline
	Bilen skal have en maksimal vægt på 2 kg & Bilen placeres på en vægt & Vægten er under 2 kg & & \\ \hline
	Bilen skal måle sin fart med en opløsning på 0,5 km/t og bilen skal kunne indstille sin fart med en opløsning på 0,5 km/t & Der afmåles et vejbanestykke på 20m og et på 10m i forlængelse af hinanden. Bilen indstilles til forskellige hastigheder fra 0 km/t til 13 km/t i trin af 0,5 km/t. Idet bilen passerer de første 20m startes et stopur og stoppes idet den har bevæget sig yderligere 10m. Der aflæses fart ud fra driftsstatus i løbet af de sidste 10m. & Bilen kan aflæse og indstille sin fart med en opløsning på 0,5 km/t & & \\ \hline 
\end{tabularx}








