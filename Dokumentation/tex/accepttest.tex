\chapter{Accepttest}

Accepttesten udføres ved brug af en skaleret tosporet motorvej, i samme størrelsesforhold som bilen 1:10. Den konstrueres vha. hvid tape, som repræsenterer vejbanestriberne. Vejen skal være 35cm bred, derudover skal der laves et lige vejbanestykke på minimum 20m. Efter dette stykke konstrueres vejen således, at sving svarer til forholdene på en dansk motorvej

\section{Testudstyr}
\begin{itemize}
	\item Hvid tape
	\item Papkasse i samme størrelse som bil(benævnes som testobjekt)
	\item Målebånd
	\item Vægt
\end{itemize}


%% Skabelon
%\begin{table}[H] 			
%	\centering
%	\begin{tabularx}{\textwidth}{|c|X|X|X|X|}
%		\hline
%		\bfseries Use case under test & \multicolumn{4}{|c|}{< Use case >} \\ \hline
%		\bfseries Scenarie & \multicolumn{4}{| c |}{< Scenarie >} \\ \hline
%		\bfseries Prækondition &  \multicolumn{4}{|c|}{< Prækondition >} \\  \hline
%		\bfseries Step  & \bfseries Handling &  \bfseries Forventet & \bfseries Faktisk & \bfseries Vurdering \\ \hline 
%		\end{tabularx}
%		\caption{< Caption >}
%\end{table}


\subsection{Test af ikke-funktionelle krav}

\begin{tabularx}{\textwidth}{|X|X|X|X|X|}
	\hline
	\textbf{Krav} & \textbf{Test} & \textbf{Forventet resultat} & \textbf{Resultat} & \textbf{Vurdering} \\ \hline
	Bilens længde og bredde må ikke overskride 50cm x 30cm & Længde og bredde måles & Længden og bredden overskrider ikke 50cm x 30cm & & \\ \hline
	Brugeren skal have mulighed for at kommunikere med bilen via tekstterminal og / eller grafisk brugergrænseflade & & & & \\ \hline
	Bilens skal detektere objekter på en afstand i intervallet 20cm til 2,5m i spring af 5cm & Der placeres et objekt 20cm foran bilen. Driftsstatus vælges og aflæses via interface. Objektet føres ud i en afstand 2,5m i skridt af 5cm og driftsstatus vælges og aflæses via interface. & Detektion i intervallet fra 20cm til 2,5m med opløsning på 5cm & & \\ \hline
	Bilen skal kunne køre med en fart på mindst 13 km/t $\pm$ 0,5 & Der afmåles et vejbanestykke på 20m og et på 10m i forlængelse af hinanden. Der angives en fart på 13 km/t. Idet bilen passerer de første 20m startes et stopur og stoppes efter bilen har bevæget sig yderligere 10m  & Bilen kan køre 13 km/t $\pm$ 0,5 km/t & & \\ \hline
\end{tabularx}



\begin{tabularx}{\textwidth}{|X|X|X|X|X|}
	\hline
	\textbf{Krav} & \textbf{Test} & \textbf{Forventet resultat} & \textbf{Resultat} & \textbf{Vurdering} \\ \hline
	Bilen skal have en maksimal vægt på 2 kg & Bilen placeres på en vægt & Vægten er under 2 kg & & \\ \hline
	Bilen skal måle sin fart med en opløsning på 0,5 km/t og bilen skal kunne indstille sin fart med en opløsning på 0,5 km/t & Der afmåles et vejbanestykke på 20m og et på 10m i forlængelse af hinanden. Bilen indstilles til forskellige hastigheder fra 0 km/t til 13 km/t i trin af 0,5 km/t. Idet bilen passerer de første 20m startes et stopur og stoppes idet den har bevæget sig yderligere 10m. Der aflæses fart ud fra driftsstatus i løbet af de sidste 10m. & Bilen kan aflæse og indstille sin fart med en opløsning på 0,5 km/t & & \\ \hline 
\end{tabularx}








