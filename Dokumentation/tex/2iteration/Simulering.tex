\section{Simulering}

I dette afsnit laves simuleringen for det samlede kredsløb i 2. iteration. 
Selve simuleringsdokumentet er delt op i blokke for at gøre det mere overskueligt. 


\noindent Kigges der på det yderste trin på figur~\ref{fig: simtop}, ses blot indgangsspændingen på 26V og udgangsloaden, der er sat op til $8.4\ohm$. Loaden er sat op som en strømkilde, der vil trække en strøm på $2.5A$ fra converteren.
\begin{figure}[H]
	\center
	\includegraphics[max width=0.7\linewidth]{/tex/2iteration/billeder/Simulering_2iteration_top.png}
	\caption{Yderste blok af simulering}
	\label{fig: simtop}
\end{figure}
Imellem er blokken "Flyback". Heri er selve kredsløbet. Dykkes der ind i denne blok fås det der ses på figur~\ref{fig: simfly}. 
\begin{figure}[H]
	\center
	\includegraphics[max width=0.9\linewidth]{/tex/2iteration/billeder/Simulering_2iteration_flyback.png}
	\caption{Flyback blok}
	\label{fig: simfly}
\end{figure}
Her ses yderligere 2 blokke hhv. Inputfilter og flyback converter. Ud over disse blokke ses de komponenter, der er brugt til at få PWM controlleren til at køre efter hensigten. Selve PWM-controlleren ligger inde i flyback converter blokken. Værdierne og forklaringen af komponenterne blev gennemgået i analyse afsnittet om PWM controlleren.
Desuden ses output kondensatoren med de udregnede parasitter også.


\noindent Blokken for inputfiltret er allerede vist tidligere under forklaringen af denne, så den vises ikke igen. Til gengæld ses indholdet af Flyback converter blokken på figur~\ref{fig: simflycon}. 
\begin{figure}[H]
	\center
	\includegraphics[max width=0.9\linewidth]{/tex/2iteration/billeder/Simulering_2iteration_flycon.png}
	\caption{Flyback converter blok}
	\label{fig: simflycon}
\end{figure}
Heri ses selve PWM controlleren UCC1801, som der er trukket en model ind for\cite{ucc1801-pspice}. Også MOSFET'en og Dioden er der trukket modeller ind for. Ved MOSFET'en har det ikke været muligt at finde den præcise model. Derfor er IRF630 modellen istedet brugt, da det er vurderet, at den minder en del om den\cite{IRF630}. 
Yderligere ses transformatoren, hvor både spredningsselvinduktion og kobbermodstanden i ledningerne er tegnet med samt kernemodellen for 3F3 er trukket ind. Derudover er koblingen i transformatoren regnet til $99.73\percent$, ud fra forholdet mellem selvinduktionen og spredningsselvinduktionen.


\subsection{Constant load} \label{constant}
Ved constant load simuleringen simuleres  der med en load på $8.4\ohm$, efter $20ms$ så det sikres, at der ses på den stationære udgang. Indgangsspændingen er sat til 26V.
Første plot af denne simulering ses på figur~\ref{fig: simflycon}. Her ses både strøm og spænding på udgangen.
\begin{figure}[H]
	\center
	\includegraphics[max width=0.9\linewidth]{/tex/2iteration/billeder/Simudgang.png}
	\caption{Simulering af udgang}
	\label{fig: simudgang}
\end{figure}
Her ses det at spændingen V(out) ligger på 21V, dog med svingninger hver gang der switches. Det ser altså ud til at switching transienter fra MOSFET og diode kommer til syne på udgangen. Det er samme billede for strømmen I(Gload), der ellers ligger på de forventede 2.5A.

\noindent På figur~\ref{fig: simMOSdio} ses en spændingsperiode for drain benet på MOSFET'en samt dioden. 
\begin{figure}[H]
	\center
	\includegraphics[max width=0.9\linewidth]{/tex/2iteration/billeder/SIMMOSFETdiode.png}
	\caption{Simulering af spænding over diode og drain ben på MOSFET}
	\label{fig: simMOSdio}
\end{figure}
Det ses, at når transistoren (rød) går OFF så kommer den tidligere omtalte peakspænding samt den svinger, inden den falder mod en stationær værdi på ca. 48V, inden MOSFET'en switches ON igen. Dette stemmer fint overens med analysen hvor den stationær værdi bør ligge på $21V+26V=47V$. Den ekstra spænding over MOSFET'en, er bl.a. et bidrag fra spændingsfaldet over dioden.
Peak'en aflæses ca. 93V.
Det samme ses for dioden (grøn) at når MOSFET'en er ON, vil dioden forspændt i spærreretningen, og skal derfor kunne holde til den peak på ca. 80V der ses på grafen. Derudover lægger den sig på en stationær værdi på ca. 46V, hvilket igen stemmer pænt overens med de 47V. 

\noindent På figur~\ref{fig: simMOSzoom} zoomes der ind på svingningerne på MOSFET'ens drain.
\begin{figure}[H]
	\center
	\includegraphics[max width=0.9\linewidth]{/tex/2iteration/billeder/Simulering_MOSFET_drain_zoom.png}
	\caption{Zoomet simulering af svingninger fra MOSFET}
	\label{fig: simMOSzoom}
\end{figure}
Her kan svingningernes frekvens aflæses. Det gøres ved, at aflæse længden på en svingning. Her er anden svingning aflæst til 34ns. Det giver en frekvens på 29.41MHz.

\noindent Det samme gøres for det zoomede billede af svingningerne i dioden som ses på figur~\ref{fig: simdiodezoom}. 
\begin{figure}[H]
	\center
	\includegraphics[max width=0.7\linewidth]{/tex/2iteration/billeder/Simulering_diode_anode_zoom.png}
	\caption{Zoomet simulering af svingninger fra diode}
	\label{fig: simdiodezoom}
\end{figure}
\noindent Anden svingning er her aflæst til 30ns. Det giver en frekvens på $33.33M\hertz$.

%%% Simulering af PWM-controller for 2. iteration

\subsection{PWM-controller}
I det følgende afsnit simuleres funktionaliteterne omkring PWM-controlleren. Her simuleres frekvensen af savtandspændingen og selve switch-frekvensen, samt signalet over current-sense modstanden både før og efter filteret.

\subsubsection{Switch-frekvens}
\noindent Først simuleres frekvensen af savtandspændingen. Dette er gjort på figur~\ref{fig:Simulering_PWM_savtand}. Periodetiden af savtandspændingen aflæses til $5.01\micro s$. Omregnet til en frekvens giver det: $f_{osc}=\frac{1}{5.01\micro s}=199.6k\hertz$. Derudover aflæses signalets minimumsspænding til ca. $138mV$, og maksimum- til ca. $2.5V$. I følge teorien burde spændingen ligge mellem $200mV$ og $2.65V$.

\begin{figure}[H]
	\center
	\includegraphics[max width=0.7\linewidth]{/tex/2iteration/billeder/Simulering_PWM_savtand.png}
	\caption{Simulering af savtandspændingen}
	\label{fig:Simulering_PWM_savtand}
\end{figure}

\noindent Nu simuleres selve switch-frekvensen. Dette er gjort på figur~\ref{fig:Simulering_PWM_switch_frekvens}, hvor der måles på udgangen af PWM-controlleren. Her måles periodetiden til $10.1\micro s$, eller en frekvens på $f_s=99.01k\hertz$. 

\begin{figure}[H]
	\center
	\includegraphics[max width=0.7\linewidth]{/tex/2iteration/billeder/Simulering_PWM_switch_frekvens.png}
	\caption{Simulering af switch-frekvens}
	\label{fig:Simulering_PWM_switch_frekvens}
\end{figure}

\subsubsection{Switch-tid}
\noindent Simuleringen af switch-tiden i MOSFET'en er vist på figur~\ref{fig:Simulering_MOSFET_switch_tid}. Figuren viser et zoom af MOSFET'ens gate signal. Switch-tiden kan aflæses som tiden af signalets plateau. Det aflæses til ca. $103ns$. Grunden til dette afviger meget fra analysen, er modellen for den MOSFET der bruges i simuleringen. Som nævnt bruges en anden MOSFET i simuleringen, end der er regnet med i analysen. En af de specifikationer de afviger mellem de to MOSFETs er \textit{Miller} ladningen, der bruges til at regne switch-tiden. I databladet for IRF630\cite{IRF630}, aflæses den til $15nC$. Regnes switch-tiden ud fra det, fås $107.1ns$, hvilket stemmer med det simulerede. 
\begin{equation} 
T_{ch} = \frac{Q_{gd} \cdot R_{g}}{V_{DD}-V_{gs}} = \frac{15nC \cdot 50\ohm}{12V-5V} = 107.1ns
\end{equation}

\begin{figure}[H]
	\center
	\includegraphics[max width=0.7\linewidth]{/tex/2iteration/billeder/Simulering_MOSFET_switch_tid.png}
	\caption{Simulering af switch-tid i MOSFET}
	\label{fig:Simulering_MOSFET_switch_tid}
\end{figure}


\subsubsection{Current-sense}
\noindent Current-sense signalet måles både før og efter filtreringen. Signalet før filteret ses på figur~\ref{fig:Simulering_PWM_current_sense_U}. Her ses de spikes på signalet, der kan give anledning til en forkert duty-cycle. Figur~\ref{fig:Simulering_PWM_current_sense_M} viser simuleringen for signalet efter filtreringen. Her ses det, at spikesene er blevet filtreret væk, dog med den konsekvens at signalet har fået en langsommere stigetid. Stigetiden aflæses til ca. $280ns$. Ifølge teorien burde PWM-controllerens udgangssignal skifte, når current-sense signalet er lig $1V$. På figur~\ref{fig:Simulering_PWM_current_sense_M}, ses det, at dette også er tilfældet for p-spice modellen for UCC1801. 


\begin{figure}[H]
	\center
	\includegraphics[max width=0.7\linewidth]{/tex/2iteration/billeder/Simulering_PWM_current_sense_U.png}
	\caption{Simulering af current-sense signal før filtrering}
	\label{fig:Simulering_PWM_current_sense_U}
\end{figure}

\begin{figure}[H]
	\center
	\includegraphics[max width=0.7\linewidth]{/tex/2iteration/billeder/Simulering_PWM_current_sense_M.png}
	\caption{Simulering af current-sense signal efter filtrering}
	\label{fig:Simulering_PWM_current_sense_M}
\end{figure}




%%%%%%%%%%%%%%%%%%%%%%%%%%%%%%%%%%%%%%%%%%%%%%%
%%% Simulering af gain-fase for 2. iteration

\subsection{Spændingsdeler}
Simulering af spændingsdeleren ses på figur~\ref{fig:simulering_voltage_divider}. Her er der målt et gennemsnit af den spænding der ligger over $R_{FB2}$, og dermed den spænding der ligger på indgangen af fejlforstærkeren. Spændingen aflæses til $2.499V$, hvilket blev designet efter $2.5V$.


\begin{figure}[H]
	\center
	\includegraphics[max width=0.7\linewidth]{/tex/2iteration/billeder/Simulering_voltage_divider.png}
	\caption{Simulering spændingsdeler}
	\label{fig:simulering_voltage_divider}
\end{figure}


\subsection{Gain-fase}
Simuleringsdiagrammet for gain-fase, ses på figur~\ref{fig:sim_gain_fase_top}. For at lave simuleringen, indføres der et fejlsignal i tilbagekoblingen til fejlforstærkeren. Det gøres ved at indsætte en ekstra modstand inden spændingsdeleren, som ses på figur~\ref{fig:sim_gain_fase_2iteration}. Impedansen af denne modstand vælges til $51.1\ohm$, for at tilpasning til den network analyzer der bruges i realiseringen. Derudover vil modstanden ikke påvirke spændingsdeleren, da den er meget lille ift. $RFB1$. Da der indføres et fejlsignal på indgangen, mens ændringen på udgangen monitoreres, er det åbensløjfen der simuleres. 

\begin{figure}[H]
	\center
	\includegraphics[max width=0.7\linewidth]{/tex/2iteration/billeder/Simulering_gain_fase_top.png}
	\caption{Gain-fase simuleringsdiagram}
	\label{fig:sim_gain_fase_top}
\end{figure}

\begin{figure}[H]
	\center
	\includegraphics[max width=0.7\linewidth]{/tex/2iteration/billeder/Simulering_gain_fase_2iteration_flyback.png}
	\caption{Flyback blok - Gain-fase simulering}
	\label{fig:sim_gain_fase_2iteration}
\end{figure}

For at lave gain-fase simuleringen skal der foretages et frekvens-sweep af fejlsignalet. Derfor skal der laves en AC-simulering af kredsløbet. Det er dog ikke muligt med det bibliotek der bruges af UCC1801. Derfor foretages der en transient simulering, hvor frekvensen af fejlkilden løbende hæves. Dette giver en fil, med en simulering af alle ønskede frekvenser. Ud fra denne fil kan p-spice generere den tilhørende AC-fil, hvor gain-fase af systemet kan plottes. 

For at sætte simuleringen op i p-spice, indsættes tekstblokken \textit{@PSpice:}. Den ses på figur~\ref{fig:sim_gain_fase_top}. I den vælges længden på simuleringerne og frekvens-sweep'et. På første linje defineres længden af hver simulering. Den første og den sidste($0$ og $SSD$) del af linjen definerer opstarten, hvor $SSD=15ms$. Ved kommandoen \textit{Skipbp}, vælges det at se bort fra opstarten, og dermed kun se steady-state. Den midterste del af linjen($5/Freq+SSD$) definerer hvor langt selve signalet skal være. Hvor Freq er den variable frekvens. $5/Freq$ bestemmer at der simuleres fem perioder, og $+SSD$ tager højde for at opstarten ikke medtages. 

Anden linje definerer om der ønskes et lineært eller et logaritmisk sweep. Ved at bruge kommandoen \textit{dec}, for deacde, vælges et logaritmisk sweep. Derudover vælges start- og stopfrekvens, samt hvor mange simuleringer der skal foretages i hver dekade. Her vælges en startfrekvens på $100\hertz$, en slutfrekvens på $100k\hertz$, og 10 simuleringer per dekade. Da systemets dominerende pol ligger ved $132\hertz$, ville det være optimalt at have en startfrekvens på $10\hertz$. Dette var dog ikke muligt pga. simuleringstid og filstørrelse. De sidste to linjer fortæller p-spice, at det er Freq der er variabel. 

Opsætningen af indgangskilden og udgangsloaden, gøres på samme måde som ved Constant Load. Amplituden af fejlkilden, opsættes således amplituden falder når frekvensen stiger. Dette sikrer at systemet ikke overstyres ved høje frekvenser. Samtidig sikres på den måde at have en lidt højere amplitude ved de lavere frekvenser. Det hjælper til signalstøjforholdet, som er mest kritisk ved de lavere frekvenser, da gain her er høj\cite{evalgp}.

Simulering af gain-fase for selve power modulet er vist på figur~\ref{fig:sim_gain_fase_power}. Her er gain grøn og fasen er rød. Der er målt mellem udgangen af fejlforstærkeren og udgangen af converteren. Det ses at når frekvensen nærmer sig $10k\hertz$, bliver simuleringen usikker. Derfor vil det primært være analysen der bruges til sammenligning med realiseringen. Båndbredden kan dog aflæses til ca. $1200\hertz$, hvilket stemmer nogenlunde overens med analysen, som blev aflæst til ca. $1430\hertz$. Derudover aflæses gain ved en frekvens på $100\hertz$ til ca. $17.4\decibel$. Ved analysen blev dette aflæst til ca. $18\decibel$. Ud fra disse punkter kan det ses at simulering passer nogenlunde ved de lavere frekvenser. 

\begin{figure}[H]
	\center
	\includegraphics[max width=0.7\linewidth]{/tex/2iteration/billeder/Simulering_gain_fase_power.png}
	\caption{Gain-fase simulering for power modul}
	\label{fig:sim_gain_fase_power}
\end{figure}

\noindent Simuleringen af gain-fase for fejlforstærkeren er vist på figur~\ref{fig:sim_gain_fase_2iteration_opamp}. Der er målt mellem indgangen til spændingsdeleren og udgangen af fejlforstærkeren. Det ses, at den ønskede funktionalitet af fejlforstærkeren er opnået. Ved frekvenser over knækfrekvensen på ca. $300\hertz$, har fejlforstærkeren et konstant gain på ca. $-5.3\decibel$. Mens forstærkningen stiger ved lavere frekvenser. Derudover ses det at fejlforstærkeren tilfører et faseløft på $90^\circ$, da der indføres en pol.

\begin{figure}[H]
	\center
	\includegraphics[max width=0.7\linewidth]{/tex/2iteration/billeder/Simulering_gain_fase_2iteration_opamp.png}
	\caption{Gain-fase simulering for fejlforstærker}
	\label{fig:sim_gain_fase_2iteration_opamp}
\end{figure}


\noindent Simuleringen af gain-fase for det samlede system er vist på figur~\ref{fig:sim_gain_fase_2iteration_tot}. Der er målt over den modstand hvor fejlsignalet indføres. Det svarer også til indgangen af spændingsdeleren og udgangen af converteren. Her ses det samme, som ved power modulet, at simuleringen ved de høje frekvenser bliver usikker. Indførelsen af fejlforstærkeren ses, ved båndbredden er blevet begrænset til ca. $700\hertz$. Der er designet efter en båndbredde på ca. $800\hertz$. Gain-marginen kan ikke rigtig aflæses her, da simuleringen bliver ustabil inden fasen krydser $0^\circ$. Til gengæld kan fase-marginen aflæses til ca. $75^\circ$, hvilket i analysen blev designet til $74.3^\circ$. 

\begin{figure}[H]
	\center
	\includegraphics[max width=0.7\linewidth]{/tex/2iteration/billeder/Simulering_gain_fase_2iteration_tot.png}
	\caption{Gain-fase simulering for det samlede system}
	\label{fig:sim_gain_fase_2iteration_tot}
\end{figure}




  

\subsection{Load-step} \label{loadstep2ite}
Ved load-steppet kontrolleres det, hvor hurtigt systemet får reguleret udgangsspændingen, efter en pludselig ændring af loaden. I forhold til før, er der sket en ændring i toplaget i schematic, som ses på figur~\ref{fig: simloadtop}  
\begin{figure}[H]
	\center
	\includegraphics[max width=0.7\linewidth]{/tex/2iteration/billeder/Simloadtop.png}
	\caption{Toplaget for simulering af load step}
	\label{fig: simloadtop}
\end{figure}
Den eneste forskel er i udgangsloaden. Som her består af to $20\ohm$ modstande i parallel, hvor den ene sidder for enden af 2 switches. Switchen U1 lukker efter $6ms$ hvorefter begge switches vil være ON. Her er loaden parallelmodstanden af Rload1 og Rload2 og dermed $10\ohm$. Dette er tilfældet i 4ms, indtil switch U2 åbner og loaden igen består af de $20\ohm$ fra Rload1.

Udover dette er der en yderligere ændring. Ved denne simulering er parasit spolen fjernet fra udgangskondensatoren. Dette er gjort, da spolen som før set, giver anledning til en del svingninger. Det er ikke den del, der her i load steppet er interessant, og derfor ses der bort fra den. Det vigtige er i stedet hvor hurtigt der bliver reguleret og hvor stort et overshoot der fås. Simuleringsresultatet ses på figur~\ref{fig: simloadstep} 

\begin{figure}[H]
	\center
	\includegraphics[max width=0.9\linewidth]{/tex/2iteration/billeder/Load_step_simulering.png}
	\caption{Simulering af load-step}
	\label{fig: simloadstep}
\end{figure}

Det ses, at så snart switch U1 går ON falder spændingen fra $21V$ til ca. $20.35V$ og det tager systemet ca. 1.6ms at regulere tilbage igen. Da U2 går OFF stiger spændingen fra 21V til 21.7V og bruger igen 1.6ms, på at regulere spændingen tilbage til de 21V. 



\subsection{Tab}
Her er tabene for de forskellige komponenter simuleret, ligesom de blev analyseret tidligere. Her er der igen givet et overblik over det samlede tab til sidst i afsnittet.

\subsubsection{Transformator}
Kernetabet i transformatoren er simuleret tidligere i sektion ~\ref{Simkernetab} til $311m\watt$. Der er indsat modstande på $53.33\ohm$ på både primær- og sekundærsiden af transformatoren for at simulere kobbertabet i trådene. På figur~\ref{fig: Simkobbertab} ses det simulerede kobbertab. Her er der taget et gennemsnit af effektafsættelserne i de to kobbermodstande lagt sammen. 
\begin{figure}[H]
	\center
	\includegraphics[max width=0.9\linewidth]{/tex/2iteration/billeder/Simkobbertab.png}
	\caption{Simulering af kobbertab i transformator}
	\label{fig: Simkobbertab}
\end{figure}

Tabet aflæses til $1.31\watt$. Det ekstra tab der er simuleret i forhold til analysen kommer af, at current-sense modstanden er designet efter en peakstrøm på ca. 6A. Det får RMS strømmen på primær og sekundær til at stige, og ligger dermed på et højere niveau end den analyserede værdi.

\subsubsection{MOSFET} \label{MOSFETsimtab}
\noindent Conduction- og switch-tabene er her simuleret. På figur~\ref{fig: SimMOStab} ses tabet for hver periode.

\begin{figure}[H]
	\center
	\includegraphics[max width=0.9\linewidth]{/tex/2iteration/billeder/SimMOStab.png}
	\caption{Simulering af tab i MOSFET (periodeoverblik)}
	\label{fig: SimMOStab}
\end{figure} 

Det ses, at der kommer de førnævnte effekttrekanter når der switches. Det er det simulerede switchtab. Offsettet i tiden mellem disse effekttrekanter er conduction tabet.
På figur~\ref{fig: simloadstep} er der taget et gennemsnit af simuleringen. Det giver det samlede tab for både conduction og switch-tabet. 
\begin{figure}[H]
	\center
	\includegraphics[max width=0.9\linewidth]{/tex/2iteration/billeder/SimMOSavg.png}
	\caption{Simulering af average tab i MOSFET}
	\label{fig: SimMOSavg}
\end{figure} 

Det aflæses til at ligge på 4.45W. Tabet her ligger en del  under det analyserede. Det skyldes, at modellen der er brugt i simuleringen, ikke er den samme som bruges i analyse og realisering. Samtidig er analysen et estimat af tabet, hvor der foreksempel regnes med at effekttrekanterne er lige store. Begge dele kan være fejlkilder der gør, at de to tab ikke stemmer helt overens.  

\subsubsection{Diode}
\noindent Diodens tab er simuleret på figur~\ref{fig: simdiodetab}. 
\begin{figure}[H]
	\center
	\includegraphics[max width=0.9\linewidth]{/tex/2iteration/billeder/Simdiodetab.png}
	\caption{Simulering af tab i diode}
	\label{fig: simdiodetab}
\end{figure}
Tabet aflæses til $1.47W$. Grunden til det ikke stemmer så godt overens med det analyserede, skyldes at modellen regner med et spændingsfald på $0.6V$, hvor der i analysen er brugt $0.45V$. Dette kommer af, at simuleringsmodellen ikke tager højde for at spændingsfaldet ændrer sig med strømmen og temperaturen. 

\subsubsection{CS modstands tab}
\noindent Tabet i current sense modstanden ses på figur~\ref{fig: simdCStab}
\begin{figure}[H]
	\center
	\includegraphics[max width=0.9\linewidth]{/tex/2iteration/billeder/SimCStab.png}
	\caption{Simulering af tab current sense modstand}
	\label{fig: simdCStab}
\end{figure}
Tabet kan aflæses til $2.03W$. Igen er tabet noget større end beregnet, og skyldes den forstørrede RMS strøm der løber i modstanden.

\subsubsection{Samlet tab}
\begin{table}[H] 			
	\centering
	\begin{tabularx}{\textwidth}{|X|l|l|}
		\hline
		\textbf{\large Komponent} & \multicolumn{2}{|l|}{\textbf{\large Tab}} \\ \hline
		& A & S	\\ \hline
		\textbf{Transformator samlet} & $1.46\watt$ & $1.62\watt$ \\ \hline 
		Kernetab & $366m\watt$ & $311m\watt$ \\ \hline
		Kobbertab & $1.09\watt$ & $1.31\watt$ \\ \hline
		& &	\\ \hline
		\textbf{MOSFET samlet} & $5.55\watt$ & $4.45\watt$ \\ \hline
		Conductiontab & $1.06\watt$ & \\ \hline
		Switchtab & $4.49\watt$ & \\ \hline
		& &	\\ \hline
		\textbf{Diode} & $1.13\watt$ & $1.47\watt$ \\ \hline
		& &	\\ \hline
		\textbf{CS modstands tab} & $1.52\watt$ & $2.03\watt$ \\ \hline
		& &	\\ \hline
		\textbf{Total tab} & $9.67\watt$ & $9.57\watt$ \\ \hline
	\end{tabularx}
	\caption{Oversigt over analyseret og simuleret tab}
	\label{tab:anasim}
\end{table}

For at få det samlede simulerede tab i converteren, tages der på figur~\ref{fig: samletsimtab} et gennemsnit af den effekt der trækkes fra indgangskilden, i forhold til den belastning loaden trækker på udgangen. Differensen herimellem er det samlede simulerede tab.
\begin{figure}[H]
 	\center
 	\includegraphics[max width=0.9\linewidth]{/tex/2iteration/billeder/Samletsimtab.png}
 	\caption{Simulering af watt på indgangen og i loaden}
 	\label{fig: samletsimtab}
\end{figure}

Indgangseffekten aflæses til ca. $62.75\watt$ og loadens effekt til $53.25\watt$. Det giver et samlet simuleret tab på $9.5\watt$. Det minder meget om resultatet i tabel ~\ref{tab:anasim}. Det betyder, at der er taget højde for de vigtigste tab i converteren. 

