\section{Indgangsfilter}
Indgangsfiltreret har som sådan ikke været en del af projektet, da det er blevet givet af Terma. Dette blev gjort, så der kunne fokuseres andetsteds. 

\noindent Herunder beskrives filtret dog stadig, så der gives forståelse for, hvordan det er designet og hvad det bruges til.

Et input filter er nødvendigt i n SMPS. For det første skal det sikre imod den elektromagnetiske interferens(EMI), der bliver genereret fra switching. Hvis denne interferens får lov, at komme ud på forsyningsnetværket, vil det påvirke andet udstyr, hvilket selvfølgelig ikke er hensigtsmæssigt. Mængden af tilladeligt EMI er fastlagt af standarder verden over, så hvis ikke produktet overholder dette, kommer det aldrig på markedet\cite{inputfilter}.

Udover EMI skal filtret også sikre, at højfrekvent spænding fra forsyningsnettet, ikke når outputtet for converteren. 

Det er i dette projekt gjort med et parallel dæmpet filter. Det består af et LC led, der giver en overordnet resonans frekvens. I sig selv, bør det kunne sikre sig imod de 2 punkter beskrevet ovenfor. Problemet ved kun at bruge denne del, ligger ved knækfrekvensen for filteret. Når filtret er udæmpet, som det er ved et LC, kan der komme stor forstærkning ved cut-off frekvensen, og derfor forstærke støjen ved denne frekvens. Det betyder at der gerne skal ligge en stor dæmpning ved frekvensen. Med en lille dæmpningsfaktor fås et stort gain ved cut-off frekvensen og omvendt. En dårlig dæmpningsfaktor kan give resten af systemet en dårligere performance. Det kan gå ind og påvirke overføringsfunktionen til reguleringsloopet og på den måde få systemet til at oscillere. Hvis der sørges for, at udgangsimpedans kurven for indgangsfiltret ligger meget under impedanskurven for konverteren, vil konverterens loop gain ikke blive ændret det store. Det vil sige, at det er vigtigt at holde peak impedansen nede for filtret, for at undgå oscillerings problemer forårsaget af inputfiltret. Figur~\ref{fig:Inputfilter_bodeplot} viser bode plots for et udæmpet indgangsfilter(rød), og et parallelt dæmpet indgangsfilter(blå). Her ses det ved brug af et parallelt dæmpet filter, at man kan filtrere peak'en fra det udæmpede filter væk. hvilket sikre en høj dæmpningsfaktor. 

\begin{figure}[H]
	\center
	\includegraphics[max width=0.7\linewidth]{/tex/2iteration/billeder/Inputfilter_bodeplot.png}
	\caption{Overføringsfunktioner for et udæmpet- og et parallelt dæmpet filter}
	\label{fig:Inputfilter_bodeplot}
\end{figure}    

Det er her, det parallelle led kommer ind i billedet. Det består af en modstand i serie med en kondensator. Meningen med modstanden er, at reducere udgangs peak impedancen af filtrets cutoff frekvens. Samtidig vil kondensatoren i serie med modstanden sørge for, at blokere DC delen af inputspændingen og derfor mindske effekttabet i modstanden. Denne kondensator skal have en mindre impedans end modstanden ved resonans frekvensen og en større kapacitet end filter kapaciteten. Dette vil gøre, at cut-off frekvensen af R-L filtret ikke påvirkes af kondensatoren. På figur~\ref{fig: Inputfilter} ses hele filteret givet af Terma.    
\begin{figure}[H]
	\center
	\includegraphics[max width=0.7\linewidth]{/tex/2iteration/billeder/Inputfilter.png}
	\caption{Inputfilter}
	\label{fig: Inputfilter}
\end{figure}

\noindent Det ses, at filtret har en overordnet knækfrekvens på:
\begin{equation} \label{fc}
f_c = \frac{1}{2 \cdot \pi \cdot \sqrt{3\micro H \cdot 10\micro F}} = 29.06kHz
\end{equation}
Samtidig kan det konkluderes, at selve kapaciteten på $C_{in2}$ er 4 gange større end $C_{in}$ samt impedansen bliver:
\begin{equation} \label{XC2}
X_{Cin2} = \frac{1}{2 \cdot \pi \cdot 40\micro F \cdot 29.06kHz} = 0.137\ohm
\end{equation}
Hvilket er mindre end modstanden på $0.8\ohm$. Det vil sige, at filtret opfylder kriterier opstillet ovenfor.