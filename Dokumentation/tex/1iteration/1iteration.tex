% Dokumentation af converter topologi

\chapter{Første Iteration}
I dette afsnit beskrives den indledende og første iteration af designfasen. Den indebærer valg af converter topologi, samt simulering af en ideel converter.

\section{Switch Mode Power-Supply}
I dette projekt vælges der at tage udgangspunkt i Switch Mode Power-Supply (SMPS). Da der er stillet et krav om et maksimalt tab på 5W, betyder det, ved en maksimal udgangseffekt på 75W, at converteren skal have en effektivitet på:
\begin{equation}
	\eta = \frac{75W}{75W + 5W} \cdot 100 = 93.75\percent
\end{equation}

I en lineær converter vil effektiviteten falde hvis udgangsspænding i forhold til indgangsspænding er lav. Med udgangsspænding på 15V ud og 50V ind vil forholdet være 0.3. Det kan aflæses til en effektivitet på maksimum $30\percent$ på ~\ref{fig:linear_regulator_eff}. Da dette ikke vil kunne efterleve kravet på $93.75\percent$, udelukkes de lineære convertere. Dette kan til gengæld tilnærmes ved brug af en SMPS. Ved optimering af tabene i converteren, kan man opnå en effektivitet på mere end $90\percent$\cite{linear-regulator}.

\begin{figure}[H]
	\center
	\includegraphics[max width=0.7\linewidth]{/tex/1iteration/billeder/linear_regulator_eff.PNG}
	\caption{Maksimal effektivitet fo lineære regulatorer}
	\label{fig:linear_regulator_eff}
\end{figure}

\section{Buck Converter}
En simpel converter der bruges til nedregulering af en spænding, er buck converteren. Den består af en transistor, der er placeret i serie med et lavpas filter, i form af et LC-filter. Derudover er der placeret en diode før filteret, således strømmen i spolen har en løbevej, når transistoren går OFF. Det overordnede kredsløb for en buck converter er vist på figur~\ref{fig:buck_converter_circuit}.


\begin{figure}[H]
	\center
	\includegraphics[max width=0.7\linewidth]{/tex/1iteration/billeder/Buck_converter_circuit.PNG}
	\caption{Ideelt diagram af buck converteren
		\cite{buck-converter}}
	\label{fig:buck_converter_circuit}
\end{figure}


I transistorens ON tid, vil strømmen i spolen, og dermed også strømmen i transistoren, rampe op. Det gør den, da der er en positiv spænding over spolen. Den spænding er $V_L=V_S-V_O$. Når der er et positivt spændingsfald over spolen, vil dioden være forspændt i spærreretningen, og dermed ikke lede en strøm. Når transistoren går OFF, vil strømmen begynde at løbe gennem dioden, da strømmen i en spole ikke kan skifte momentant. Hvis dioden antages ideel, vil spændingen over spolen være lig $V_L=0-V_O$. Da dette giver et negativt spændingsfald over spolen, vil strømmen begynde at aflade i den. Strømmene er skitseret på figur~\ref{fig:buck_converter_current}. Her ses det, at der altid løber en strøm i spolen, mens den skiftes til at løbe i transistoren og dioden, afhængig af ON og OFF perioderne. 


\begin{figure}[H]
	\center
	\includegraphics[max width=0.7\linewidth]{/tex/1iteration/billeder/Buck_converter_current.PNG}
	\caption{Buck converter strømme
		\cite{buck-converter}}
	\label{fig:buck_converter_current}
\end{figure}


Da strømmen i spolen aldrig når $0A$, kaldes denne form for operation Continuous Conduction Mode, eller CCM. Overføringsfunktionen for en buck converter i CCM er\cite{SMPS-topologies2}:
\begin{equation} \label{buck_converter_overforinsfunktion}
V_{out} = D\cdot V_{in}
\end{equation}

Converteren skal kunne opretholde en outputspænding på $21V$, ved en inputspænding på $26V$. Ved at bruge overføringsfunktionen, regnes den maksimale duty-cycle til ca. $80\percent$. 

En af fordelene ved buck converteren, er at der altid løber en strøm i spolen. Dette gør, at der kan opnås en lille ripple-strøm i filteret, og derved også et mindre tab, både i spolen og kondensatoren. 
En af ulemperne, er at transistoren sidder i den positive forsyningslinje. Dette kan give komplikationer ved switching af transistoren. Hvis der vælges en p-kanals MOSFET, skal der vælges en PWM-controller der kan håndtere switching af denne. Hvis der vælges en n-kanals MOSFET, skal gate signalet være større end forsyningen, før MOSFET'en er helt ON. Dette kræver flere komponenter, og vil derfor helst undgås. På grund af dette problem undersøges der en converter topologi, hvor MOSFET'en ikke sidder i den positive forsyningslinje.


\section{Push-pull converter}
Push-pull converteren, er en transformator baseret topologi. Man deler converteren op i to dele: Primær- og sekundærsiden. Primærsiden består af to primærviklinger, og to transistorer, hvor transistoren fungerer som en switch. Sekundærsiden består af to sekundærviklinger, to dioder, et udgangsfilter i form af en spole og en kondensator, samt udgangsbelastningen. En skites af dette er vist på figur~\ref{fig:push_pull_ideal}. Der er flere ulemper ved push-pull converteren. En er det store antal komponenter ift. andre topologier. En anden er kompleksiteten i at drive den. På grund af tolerancer i komponenterne, skal converteren reguleres efter strømmen, da den ellers vil drive mod mætning. Derudover skal den valgt PWM-controller understøtte switching af to kontakter. 

\begin{figure}[H]
	\center
	\includegraphics[max width=0.9\linewidth]{/tex/1iteration/billeder/push_pull_diagram.PNG}
	\caption{Ideelt diagram af flyback converteren
		\cite{SMPS-topologies}}
	\label{fig:push_pull_ideal}
\end{figure} 

\noindent Converteren fungerer ved skiftevis, at aktivere $T_1$ og $T_2$. Når $T_1$ er ON, vil der løbe en strøm i den øverste primærvikling. Det vil transformere en spænding over på sekundærsiden, med et positivt potentiale ved $D_1$ ift. GND. Derved vil $D_1$ led den strøm der bliver transformeret over på sekundærsiden. Når $T_2$ er ON vil strømmen i stedet løbe i den nederste switch-gren, og dermed vil det også være $D_2$ der leder strømmen på sekundærsiden. I den periode en af dioderne leder en strøm vil der være et positivt potentiale over spolen på $\frac{N_s}{N_p} \cdot V_{in} - V_{out}$, og vil derfor oplade strømmen i spolen. I den periode ingen af dioderne leder en strøm, vil der være et negativt potentiale over spolen på $-V_{out}$, og derved aflade strømmen i den. Det vil generer en ripple-strøm på udgangen, hvor udgangsstrømmen vil blive set som en gennemsnitsværdi. 

Teoretisk er den maksimale duty-cycle for hver kontakt $50\percent$. I praksis skal der dog indføres en død-tid i converteren hvor begge kontakter er OFF. Dette vil sikre primærviklingen ikke bliver kortsluttet. Denne død-tid sikre samtidig afladningstiden for udgangsspolen. Både strøm og spændingskurver er vist på figur~\ref{fig:push_pull_current}. Her ses den føromtalte funktion af død-tiden i converteren. 

\begin{figure}[H]
	\center
	\includegraphics[max width=0.9\linewidth]{/tex/1iteration/billeder/push_pull_current.PNG}
	\caption{Ideelt diagram af flyback converteren
		\cite{SMPS-topologies}}
	\label{fig:push_pull_current}
\end{figure} 

\section{Flyback Converter}
Flyback converteren, er også en transformator baseret topologi. Primærsiden består af primærviklingen af transformatoren og en transistor, hvor transistoren fungerer som en switch. Sekundærsiden består af sekundærviklingen, en diode, en udgangskondensator og belastningen. Dette er vist på figur~\ref{fig:flyback_ideal}. En af fordelene ved, at bruge flyback converteren, er at der kan opnås galvanisk adskillelse mellem primær- og sekundærsiden af transformatoren, samt MOSFET'ens source er forbundet til GND. Derudover bruges der det samme antal komponenter, som ved buck converteren. Ved andre transformator baserede topologier vil antallet af komponenter være højere. 

\begin{figure}[H]
	\center
	\includegraphics[max width=0.7\linewidth]{/tex/1iteration/billeder/flyback_ideal.PNG}
	\caption{Ideelt diagram af flyback converteren
	\cite{SMPS-topologies}}
	\label{fig:flyback_ideal}
\end{figure} 

\noindent Flyback converteren bruges til, at konvertere en indgangsspænding, ned til en mindre udgangsspænding. Dette gøres ved at styre transistoren med et PWM-signal, med en variabel duty-cycle. Når den er ON, vil der være en positiv spænding ved prik-enden af viklingen ift. den anden ende. Ud fra formlen $V=L\cdot \frac{di}{dt}$ kan det ses, at når der ligger en spænding over viklingen, vil strømmen i transformatoren stige lineært, over den tid transistoren er ON. Når transistoren går OFF, vil den magnetiske strøm i transformatoren inducere en spænding over sekundærviklingen. Dette vil vende polariteten i transformatoren, således der er en prik ved henholdsvis transistoren og dioden. Nu er dioden forspændt i lederetningen, hvilket vil lade energien i transformatoren aflade ud i sekundærviklingen. Da spændingen over sekundærviklingen er positiv ved prikken, og dermed modsat af primærviklingen, vil strømmen falde lineært ud fra samme forhold, som nævnt tidligere. Dette vil over tid skabe en trekantet kurveform af den samlede strøm i transformatoren. 

Et eksempel på dette kan ses på figur \ref{fig:CCM_transformer_current}. Da strømmen i hver vikling er diskontinuert, vil det give anledning til større peak-strømme. Det er maksimalt $50\percent$ af tiden der løber en strøm i viklingen. Det giver en større peak-strøm, i forhold til buck converteren, for at kunne opretholde den samme middelstrøm.

Flyback converteren kan overordnet drives på to forskellige måder, Continuous Conduction Mode (CCM) og Discontinuous Conduction Mode (DCM). Disse to måder har forskellige fordele og ulemper, som skal tages højde for, inden der vælges hvordan converteren skal drives. 

\subsection{Continuous Conduction Mode}
Forkellen ved CCM og DCM ligger i hvordan strømmen løber i transformatoren. Ved CCM vil der altid løbe en strøm i transformatoren, som der også ligger i navnet. Dog vil strømmene individuelt i viklingerne være diskontinuerte. Strømmen er skitseret på figur \ref{fig:CCM_transformer_current}. Skal man have den samlede strøm i transformatoren, skal de to kurver for primær- og sekundærviklingen samles. Dette er fordi der kun løber en strøm i primærviklingen når transistoren er ON, og en strøm i sekundærviklingen når transistoren er OFF. 

\noindent Overføringsfunktionen for en flyback converter i CCM er\cite{SMPS-topologies2}:
\begin{equation} \label{flyback_converter_CCM_overforinsfunktion}
V_{out} = \frac{N_S}{N_P} \cdot \frac{D}{1-D} \cdot V_{in}
\end{equation}
Ud fra overføringsfunktionen ses det, at udgangsspændingen både afhænger af duty-cyclen, og af omsætningsforholdet i transformatoren. 

\begin{figure}[H]
	\center
	\includegraphics[max width=0.7\linewidth]{/tex/1iteration/billeder/CCM_transformer_current.PNG}
	\caption{CCM transformator strømme}
	\label{fig:CCM_transformer_current}
\end{figure}

\noindent En af fordelene ved CCM er, at strømmen i transformatoren ikke når at aflade helt, inden transistoren går ON igen. Dette giver lavere ripple-strømme, og dermed også peak-strømme, hvilket giver anledning til et mindre effekttab. På grund af den mindre ripple-strøm i transformatoren, opnås der også en mindre ripplespænding på udgangen. hvilket sætter et mindre krav til udgangskondensatoren. 

\subsection{Discontinuous Conduction Mode}
Den anden måde at drive converteren på er DCM. Ved denne metode vil der være en død tid i hver periode, hvor der ikke løber en strøm i transformatoren. Dette betyder at transformatoren når at aflade helt, inden switch-perioden er ovre. Til forskel fra CCM, vil dette give nogle trekantede strømkurver i transformatoren, som ses på figur~\ref{fig:DCM_transformer_current}.
På grund af død tiden, vil peak-strømmene blive større, da arealet under kurven skal være det samme som ved CCM, for at kunne opretholde den samme udgangsstrøm. 
Fordelen ved at $di$ bliver større, er at induktansen i viklingerne bliver mindre. Tilgengæld giver det anledning til større tab, da både peak- og ripple-strømmene bliver større.

\begin{figure}[H]
	\center
	\includegraphics[max width=0.7\linewidth]{/tex/1iteration/billeder/DCM_transformer_current.PNG}
	\caption{DCM transformator strømme
		\cite{SMPS-topologies}}
	\label{fig:DCM_transformer_current}
\end{figure}


\section{Ideel transformator}
Der vælges at arbejde videre med en flyback converteren, pga. komplikationerne ifm. switchingen af MOSFET'en ved buck converteren.
Der regnes strømme i transformatoren for både CCM og DCM, for derefter, at vurdere forskellene mellem de to metoder.

\noindent Der tages udgangspunkt i en converter der, ved en input spænding på $26V-50V$, skal kunne opretholde en udgang på $21V$ og $2.5A$. Derudover antages det, at transformatoren har et omsætningsforhold på 1.

\subsection{CCM} \label{maksimum_duty_cycle}
Først beregnes den maksimale og minimale duty-cycle:
\begin{equation} \label{D_maks_CCM}
D_{maks} = \frac{V_{out}}{V_{inmin} + V_{out}} = 0.447
\end{equation}
\begin{equation} \label{D_min_CCM}
D_{min} = \frac{V_{out}}{V_{inmaks} + V_{out}} = 0.296
\end{equation}

Herefter findes ripplestrømmen, som skal løbe i transformatoren. Her er der taget udgangspunkt i, at designe den efter $60\percent$ af udgangsstrømmen\cite{flyback-formler}. Dette er et tradeoff mellem størrelsen på ripplen og hvor høj en induktans der fås i viklingerne. Større induktans kræver flere vindinger og giver dermed mere tab.
\begin{equation}
I_{ripple} = 0.6 \cdot \frac{V_{out} \cdot I_{out}}{V_{inmaks} \cdot   D_{min}} = 2.13A
\end{equation}
Den nødvendige induktans det kræver for at transformatoren kan rampe op til den nødvendige strøm inden for dutycyclen, udregnes herunder\cite{flyback-formler}. Med i udregningen er switchfrekvensen, som her er valgt til $100kHz$. Da omsætningsforholdet i transformatoren er sat til 1, betyder det at $L_p = L_s$.
\begin{equation}
L = \frac{V_{inmin} \cdot D_{min}}{I_{ripple} \cdot f_s} = 69.43\micro H
\end{equation}

\noindent Med induktansen kan ripplestrømmen for minimum indgangsspænding og maksimum dutycycle findes. Det er ved denne indgangsspænding og dutycycle de maksimale peak- og RMS-strømme estimeres. 
\begin{equation} \label{I_ripple_CCM}
I_{ripple} = \frac{V_{inmin} \cdot D_{max}}{L \cdot f_s} = 1.67A
\end{equation}
Ved peak average strømmen ses strømmen i primærviklingen som en middelværdi og dermed en konstant værdi over en periode. Det er altså en estimering hvor det forventes, at der rampes lineært fra minimum strøm til peak strøm i løbet af en periode. 
\begin{equation} \label{I_pk_avg_CCM}
I_{pkavg} = \frac{I_{out}}{1-D_{maks}} = 4.52A
\end{equation}
Da ripplen er kendt, må peaken kunne findes ved at den halve ripple bliver lagt oveni den netop beregnede peak average strøm.
\begin{equation} \label{I_pk_CCM}
I_{pk} = I_{pkavg} + \frac{I_{ripple}}{2} = 5.58A
\end{equation}

\noindent Nu beregnes RMS-strømmene i både primær- og sekundærviklingerne. Her bruges peak average strømmen i anden ganget med dutycyclen\cite{transformator-design}. 
\begin{equation} \label{I_p_RMS_CCM}
I_{RMSp} = \sqrt{D_{maks} \cdot (I_{pkavg})^2} = 3.02A
\end{equation}
\begin{equation} \label{I_s_RMS_CCM}
I_{RMSs} = \sqrt{(1-D_{maks}) \cdot (I_{pkavg})^2} = 3.36A
\end{equation}

\subsection{DCM}
Nu foretages strømberegninger for en flyback converter i DCM.
Kigges der ved boundary, hvilket er det punkt hvor transformatoren lige præcis når at aflade i en switch periode, så er overføringsfunktionen for CCM og DCM den samme. Det vil sige de samme $D_{maks}$ og $D_{min}$ benyttes. 
Ved denne boundary kan peak-strømmen udregnes. Udregning af strømmen ved boundary kan estimeres til at regne i lige store trekanter. Her vil peaken være højden af trekanten og $1-D_{maks}$ er længden af trekanten på sekundærsiden. Ganges dette med en halv fås arealet af trekanten, som skal give de $2.5A$. Isoleres peak strømmen istedet i den udregning fås den til:
\begin{equation} \label{DCM_peak_current}
I_{pk} = I_o \cdot \frac{2}{1-D_{maks}} = 9.04A
\end{equation}

\noindent Da transformatoren når at aflade ved DCM, er ripple-strømmen lig peak-strømmen:
\begin{equation} \label{DCM_ripple_current}
I_{ripple} = I_{pk} = 9.04A
\end{equation}

\noindent Induktansen i primærviklingen beregnes igen ud fra ligning~\ref{L_DCM}. Da peak-strømmen, og dermed også ripple-strømmen er regnet ved boundary, betyder det, at der regnes en maksimal induktans, for hvor converteren stadig opererer i DCM.
\begin{equation} \label{L_DCM}
L = \frac{V_{inmin} \cdot D_{min}}{I_{ripple} \cdot f_s} = 12.85\micro H
\end{equation}

\noindent Da induktansen er en maksimal værdi, skal man ligge med en vis margin til denne, for at sikre, at converteren opererer i DCM. Hvis induktansen i viklingerne mindskes, vil man opnå en endnu større peak-strøm i transformatoren. Med en ripple-strøm på minimum $9.04A$, vurderes det at effekttabene ved at operere i DCM, vil blive for store. Derfor vil der fremadrettet arbejdes videre med en flyback converter i CCM.


\section{Udgangskondensator}
I en flyback converter bruges udgangskondensatoren primært til at mindske ripplespændingen på load'en. Formlen for at beregne minimumskapaciteten ses nedenfor. Her bruges kravet til 50mV pk-pk som output spændingsripple\cite{flyback-formler}.
\begin{equation} \label{udgangskondensator}
C_{out} \geqslant \frac{I_{out} \cdot D_{maks}}{V_{ripple} \cdot f_s} \geqslant 223.4 \micro F
\end{equation}

\noindent Udover at mindske ripplespændingen påvirker kondensatoren hvordan converteren reagerer på en ændring af load strømmen. Når MOSFET'en er on, vil der ikke løbe en strøm i dioden, og det er derfor kondensatoren, der skal levere strømmen til loaden. Når MOSFET'en er off, leverer kommer strømmen til loaden gennem dioden, mens den også oplader kondensatoren igen. Derudover er den, sammen med load modstanden, en del af den dominerende pol i convertertrinet. Disse egenskaber vil blive uddybet nærmere i de følgende iterationer. 

\section{Simulering}
Med udgangspunkt i figur~\ref{fig:flyback_ideal} opsættes en ideel flyback converter i p-spice. Dette er gjort på figur~\ref{fig:ideal_flyback_diagram}. Converteren er sat op med en ideel transformatorkobling, et ideelt switching element, samt en ideel diode, for at få et indblik i flyback converterens virkemåde. 


\begin{figure}[H]
	\center
	\includegraphics[max width=0.7\linewidth]{/tex/1iteration/billeder/flyback_ideal_diagram.PNG}
	\caption{Ideelt flyback kredsløb}
	\label{fig:ideal_flyback_diagram}
\end{figure}

\noindent Der er to scenarier, der er relevante at kigge på, ved en indgangsspænding på $26V$ samt en indgangsspænding på $50V$. Først kigges der på udgangen af converteren, for at kontrollere udgangsstrømmen og -spændingen. På figur~\ref{fig:26V_ideal_output} ses både outputstrømmen (rød) og -spændingen (grøn), med en inputspænding på 26V. Her ses det, at spændingen ligger sig omkring $21V$ og strømmen ligger sig omkring $2.5A$, hvilket var kravet til converteren. Derudover aflæses ripplespændingen til ca. $50mV$, hvilket overholder kravet for ripplespændingen. 

\begin{figure}[H]
	\center
	\includegraphics[max width=0.7\linewidth]{/tex/1iteration/billeder/26V_output.PNG}
	\caption{Converter output - ved 26V input}
	\label{fig:26V_ideal_output}
\end{figure}

\noindent På figur~\ref{fig:50V_ideal_output} ses det samme billede, ved $50V$ inputspænding. Da converterens duty-cycle er faldet, falder ripple-spændingen også. Den aflæses til ca. $33mV$. 


\begin{figure}[H]
	\center
	\includegraphics[max width=0.7\linewidth]{/tex/1iteration/billeder/50V_output.PNG}
	\caption{Converter output - ved 50V input}
	\label{fig:50V_ideal_output}
\end{figure}

\noindent I tabel~\ref{tab:result_ideal_converter} ses resultaterne for analyse(A) og simulering(S) af den ideelle converter. Ripple- og peak-strømmene er aflæst ud fra figur~\ref{fig:26V_transformer_current} og \ref{fig:50V_transformer_current}. RMS-strømmene findes ved, at bruge RMS-funktionen i p-spice. Derudover kan det konstateres, at converteren opererer i CCM, da transformatorstrømmen ikke når at aflade helt. Se figur~\ref{fig:CCM_transformer_current}. 

\begin{table}[H] 			
	\centering
	\begin{tabularx}{\textwidth}{|X|c|c|c|c|c|c|c|c|}
		\hline
		\textbf{Indgangs-spænding} & \multicolumn{2}{|X|}{\textbf{Ripple-strøm}} & \multicolumn{2}{|X|}{\textbf{Peak-strøm}} & \multicolumn{2}{|X|}{\textbf{RMS-strøm i primær}} & \multicolumn{2}{|X|}{\textbf{RMS-strøm i sekundær}} \\ \hline
		& A & S & A & S & A & S & A & S \\ \hline
		$26V$ & $1.67A$ & $1.66A$ & $5.36A$ & $5.35A$ & $3.02A$ & $3.08A$ & $3.36A$ & $3.33A$ \\ \hline 
		$50V$ & $2.13A$ & $2.11A$ & $4.62A$ & $4.61A$ & $1.93A$ & $1.98A$ & $2.98A$ & $3.01A$ \\ \hline
	\end{tabularx}
	\caption{Resultater for analyse og simulering af ideel flyback converter}
	\label{tab:result_ideal_converter}
\end{table}

\begin{figure}[H]
	\center
	\includegraphics[max width=0.7\linewidth]{/tex/1iteration/billeder/26V_transformer_current.PNG}
	\caption{Transformator strømme - ved 26V input}
	\label{fig:26V_transformer_current}
\end{figure}

\begin{figure}[H]
	\center
	\includegraphics[max width=0.7\linewidth]{/tex/1iteration/billeder/50V_transformer_current.PNG}
	\caption{Transformator strømme - ved 50V input}
	\label{fig:50V_transformer_current}
\end{figure}

\section{Opsummering}
Under første iteration er der undersøgt forskellige slags converter-typologier og set på fordele og ulemper for dem. Til dette projekt, er der valgt at arbejde videre med en flyback converter. Desuden er det bestemt, at converteren skal køres i CCM, da tabene mindskes grundet den mindre peak-strøm. Desuden er det beregnet og simuleret, at converteren ideelt set fungerer som forventet.   

