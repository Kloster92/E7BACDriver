%%%%%% Konklusion for tredje iteration %%%%%%

\section{Opsummering}
Det kan efter 3. iteration konkluderes, at flere funktionaliteter i converteren er blevet enten udviklet eller optimeret. I dette afsnit vil disse funktionaliteter blive opsummeret. 

\subsection{Switch-tid}
Der er opnået en switch-tid i MOSFET'en på $40ns$. Det medfører et switch-tab i den på $1.48W$, hvilket er $3W$ mindre end ved anden iteration. Samtidig er det kun $0.42W$ mere end conduction tabet i MOSFET'en, der er på $1.06W$. Dette er acceptabelt for nu og vil ikke yderligere blive optimeret. 

En konsekvens af den hurtigere switch-tid, er der kommet større spændings-peaks over MOSFET'en. De er målt til ca. $90V$, ift. ca. $80V$ ved switch-tiden i 2. iteration. Da MOSFET'en har en breakdown spænding på $150V$, accepteres denne stigning og der vil derfor ikke gøres mere ved dette.


\subsection{Current-sense filter}
Der er realiseret et lav-pas filter med en stigetid på $100ns$. Det giver et curent-sense signal med stort set rette flanker, ved både stige- og faldetid. Det vil give den mindst mulige indvirkning på converterens I/V-karakteristik. Det kan dog observeres, at signalet har et lille overshoot. Det kommer ved, at filteret kun lige akkurat filtrerer nok, for at fjerne spike'en. Da dette overshoot ikke er ret stort accepteres dette, og derfor vil det ikke blive yderligere optimeret.


\subsection{Snubber-kredsløb}
Snubber-kredsløbene er blevet realiseret for, at fjerne svingninger på MOSFET'ens drain-spænding og diodens anode-spænding. De er blevet implementeret således der kommer en enkelt svingning, og derefter ligger spændingen sig på det forventede stabile niveau. 


\subsection{Udgangsfilter}
Udgangen til loaden, blev flyttet for at udnytte de fire LC-filtre, der opstod mellem udgangskondensatorerne og ledningerne. Dette har mindsket switching-spikes på udgangen fra $10Vpk-pk$ til $900mVpk-pk$. Det opfylder stadig ikke kravet for converteren på $100mVpk-pk$, og derfor skal dette yderligere optimeres i en senere iteration. 

Det kan konkluderes ud fra impedans målingen af udgangskondensatoren, at den har en resonans frekvens ved $108k\hertz$. Det er for tæt på switch-frekvensen på $100k\hertz$, og derfor vil der blive fundet en anden kondensator i en senere iteration.

%TODO: Værdier er fra simulering, og skal ændres når målinger er udført. Det er uden 100nF kondensator


\subsection{Regulering}
Reguleringen af converteren er blevet optimeret, for at opnå en større båndbredde. Der er opstillet et krav til gain-maring på $10\decibel$ og en fasemargin på $50^\circ$, og derfor er der blevet designet efter dette. Ved en indgangsspænding på 26V, er der opnået en gain-margin på $14.5\decibel$, en fasemargin på $69.8^\circ$, og en båndbredde på $3.86k\hertz$. Derudover er der ved en indgangsspænding på 50V, opnået en gain-margin på $14.3\decibel$, en fasemargin på $76.3^\circ$, og en båndbredde på $5.7k\hertz$. Derfor kan der stadig opnås en større båndbredde, og det er noget der kan optimeres yderligere på i en senere iteration, hvis det bliver nødvendigt.


\subsection{Tab}
Ved en indgangsspænding på $26V$ blev det samlede tab i converteren målt til $5.9W$, hvilket gav en effektivitet på $90.21\percent$. Dette tab stadig for stort ift. kravene. Ud fra tabellen ses det der er tre dominerende tabsfaktorer i converteren - MOSFET, diode og current-sense modstande. Der skal overvejes optimeringsmuligheder ved disse tre komponenter for opnåelse af en større effektivitet i converteren. 


\subsection{Videreudvikling}
Funktionaliteten af converteren er blevet så tilfredsstillende, at 4. iteration vil blive udvikling af converteren således udgangen kan tilpasses til $15V$ og $5A$. Derudover skal der udvikles en $12V$ regulator til forsyning af PWM-controlleren. Tabet i converteren skal optimeres yderligere, hvor MOSFET, diode og current-sense modstande bidrager med en stor del af det samlede tab. Yderligere bør der foretages en undersøgelse af inputfiltret, da det ikke er testet eller analyseret på, mere end det Terma har givet forinden. Her tænkes på stabilitet og tab. Det vil dog ikke gøres i dette projekt pga. tidsmangel. 





