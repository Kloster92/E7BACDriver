%%%% Semesterprojekt rapport gruppe 2 %%%%
%%%% Elektronik Q1/Q2 2016 %%%%

% Kan anvendes til journaler eller afleveringer
\documentclass[11pt, a4paper, twoside, openany]{memoir}

\usepackage[utf8]{inputenc}		% Dansk input encoding (tegn)
\usepackage[danish]{babel}		% Danske formuleringer / orddeling
\usepackage[T1]{fontenc}		% Output-indkodning af tegnsaet (T1)


%%% Memoir indstillinger
%% Afstand mellem afsnit og videre
%% NIX PILLE - medmindre strengt nødvendigt
\setaftersubsubsecskip{6pt}	
\setbeforesubsubsecskip{6pt}
%\setaftersubsecskip{6pt}
%\setbeforesubsecskip{-\baselineskip}
%\setaftersecskip{6pt}
%\setbeforesecskip{-\baselineskip}
%\setaftersecskip{1ex}

\raggedbottom



\chapterstyle{section}
\usepackage{url}

% ¤¤ Marginer ¤¤ %
\setlrmarginsandblock{3.5cm}{2.5cm}{*}		% \setlrmarginsandblock{Indbinding}{Kant}{Ratio}
\setulmarginsandblock{3.0cm}{2.5cm}{*}		% \setulmarginsandblock{Top}{Bund}{Ratio}
\checkandfixthelayout

%%% Font valg %%%
\usepackage{mathpazo}	%Palatinofont - matematikformler
\usepackage{eulervm}		%Palatinofont

%%% FIGURER OG TABELLER %%%
\usepackage{graphicx} 						% Haandtering af eksterne billeder (JPG, PNG, PDF)

\usepackage[export]{adjustbox}

\usepackage{multirow}                		% Fletning af raekker og kolonner (\multicolumn og \multirow)
\usepackage{colortbl} 						% Farver i tabeller (fx \columncolor, \rowcolor og \cellcolor)
\usepackage[dvipsnames]{xcolor}				% Definer farver med \definecolor. Se mere: http://en.wikibooks.org/wiki/LaTeX/Colors
%\usepackage{flafter}						% Soerger for at floats ikke optraeder i teksten foer deres erence
\usepackage{float}							% Muliggoer eksakt placering af floats, f.eks. \begin{figure}[H]
\usepackage{multicol}         	        	% Muliggoer tekst i spalter
%\usepackage{rotating}						% Rotation af tekst med \begin{sideways}...\end{sideways}
\usepackage{booktabs}
\usepackage{bigstrut}	% Excel2latex måskeh
\usepackage{tabularx}
\usepackage{subfig}

%%% ¤¤ Matematik mm. %%%
\usepackage{amsmath,amssymb,stmaryrd} 		% Avancerede matematik-udvidelser
\usepackage{mathtools}						% Andre matematik- og tegnudvidelser
\usepackage{textcomp}                 		% Symbol-udvidelser (f.eks. promille-tegn med \textperthousand )
\usepackage{siunitx}						% Flot og konsistent praesentation af tal og enheder med \si{enhed} og \SI{tal}{enhed}
\sisetup{output-decimal-marker = {,}}		% Opsaetning af \SI (DE for komma som decimalseparator) 
\sisetup{exponent-product=\cdot, output-product=\cdot}	%Eksponent er gange tegn, output produkt er gange tegn
\sisetup{digitsep = none}					%Almindeligt komma - ingen mellemrum aka. til eurokomma

%%% REFERENCER %%%


%%% MISC %%%
\usepackage{listings}						% Placer kildekode i dokumentet med \begin{lstlisting}...\end{lstlisting}
\definecolor{bg}{HTML}{F0F0F0}
\lstset{language=C++,
				showstringspaces = false,
				backgroundcolor = \color{bg},
                basicstyle=\ttfamily,
                keywordstyle=\color{blue}\ttfamily,
                stringstyle=\color{red}\ttfamily,
                commentstyle=\color{green}\ttfamily,
                morecomment=[l][\color{magenta}]{\#},
                extendedchars=true,
                numbers=left, numberstyle=\tiny,		% Linjenumre
                columns=flexible,						% Kolonnejustering
                breaklines, breakatwhitespace=true,		% Bryd lange linjer
                literate=%
                {æ}{{\ae}}1
                {å}{{\aa}}1
                {ø}{{\o}}1
                {Æ}{{\AE}}1
                {Å}{{\AA}}1
                {Ø}{{\O}}1
}


\usepackage{lipsum}							% Dummy text \lipsum[..]
\usepackage[shortlabels]{enumitem}			% Muliggoer enkelt konfiguration af lister
\usepackage{pdfpages}						% Goer det muligt at inkludere pdf-dokumenter med kommandoen \includepdf[pages={x-y}]{fil.pdf}	
\pdfoptionpdfminorversion=6					% Muliggoer inkludering af pdf dokumenter, af version 1.6 og hoejere

%	¤¤ Afsnitsformatering ¤¤ %
%\setlength{\parindent}{0mm}           		% Stoerrelse af indryk
\setlength{\parskip}{1.5mm}          		% Afstand mellem afsnit ved brug af double Enter
\linespread{1,1}							% Linie afstand

\usepackage{tikz}


% ¤¤ Visuelle  ¤¤ %
\usepackage[colorlinks]{hyperref}			% Danner klikbare referencer (hyperlinks) i dokumentet.
\hypersetup{colorlinks = true,				% Opsaetning af farvede hyperlinks (interne links, citeringer og URL)
	linkcolor = black,
	citecolor = black,
	urlcolor = black
}



%%% Referencer / Bibliografi %%%
\usepackage[backend=biber, sorting=none, style=numeric]{biblatex}
\bibliography{../referencer.bib}


\usepackage[draft, danish]{fixme}
\fxsetup{layout=footnote}

\graphicspath{{../fig/}{../fig}{./}}


\usepackage{rotating}
\usepackage{titlesec}

\setcounter{secnumdepth}{4}

\titleformat{\paragraph}
{\normalfont\normalsize\bfseries}{\theparagraph}{1em}{}
\titlespacing*{\paragraph}
{0pt}{3.25ex plus 1ex minus .2ex}{1.5ex plus .2ex}


%%%% Opsætning af dokument %%%%
\newcommand{\forfatter}{Gruppe 2}
\newcommand{\fag}{INDSÆT KURSUS HER}
\newcommand{\titel}{Semesterprojekt 4 Dokumentation}
\date{}

\author{\forfatter}
\title{\titel}

\setlength{\beforechapskip}{10pt}
\setlength{\afterchapskip}{10pt}

\begin{document}
%\maketitle
\begin{titlingpage}
%\thispagestyle{title}
		
		\begin{center}
				{\huge\bfseries Projekt Universal Actuator Drive}\\
				\vspace{10pt}
				
				{\Huge\bfseries Dokumentation}\\
				
				\vspace{20pt}
				
				{Diplomingeniør Elektronik}\\
				{\large Bachelorprojekt efterår 2017}\\
				
				\vspace{10pt}
				
				Ingeniørhøjskolen Aarhus Universitet\\
				Vejleder: Arne Justesen
				\vspace{10pt}
				
				19. december 2017
				\vspace{10pt}

				\vspace{50pt}
				\begin{minipage}{0.25\linewidth}
					\centering
					\hrule
					\vspace{12pt}
					Nicolai H. Fransen\\
					Studienr. 201404672
				\end{minipage}
				\hspace{10pt}
				\begin{minipage}{0.25\linewidth}
					\centering
					\hrule
					\vspace{12pt}
					Jesper Kloster\\
					Studienr. 201404571
				\end{minipage}
				\hspace{10pt}
		\end{center}
		
		\clearpage
		
	\setcounter{tocdepth}{2}
	\tableofcontents
	\clearpage
	
	%\include{tex/indledning/indledning}
	
	\chapter{Kravspecifikation}

%\subsection{Prioritering / Afgrænsning} måske?
%Moscow
Kravene til produktet er prioriteret ved brug af MoSCoW metoden. Her er kravene for produktet inddelt i fire kategorier, hvor de vigtigste elementer er prioriteret højest. \textbf{Must} benævner de krav som er vigtigst at opfylde, og som er absolut nødvendigt for produktet. \textbf{Should} er de krav produktet bør opfylde. \textbf{Could} er kravene som produktet evt. kunne opfylde, hvis projektets tidsramme tillader det. \textbf{Won't} er krav som ikke vil blive opfyldt inden for projektets tidsrammer, men evt. kan tages med i senere iterationer.

\noindent Følgende opdeling viser kravene udvalgt for dette projekt:
\begin{itemize}
	\item[\textbf{Must}]
		\begin{itemize}
			\item Holde konstant udgangsstrøm og -spænding
			\item Have stabil regulering
			\item Ikke påvirke andre moduler ved fejl
			\item Konstrueres med EEE komponenter

		\end{itemize}
	\item[\textbf{Should}]
		\begin{itemize}
			\item Have programmerbar udgangsstrøm og -spænding

		\end{itemize}
	\item[\textbf{Could}] 
		\begin{itemize}
			\item Have overstrømsbeskyttelse på udgangen

		\end{itemize}
	\item[\textbf{Won't}]
		\begin{itemize}
			\item Indeholde galvanisk adskillelse
		\end{itemize}
\end{itemize}
% De krav der er kritisk for selve funktionen af bilen, dvs. bevægelse / styring, er prioriteret højest. Herefter er krav, som afhænger af bla bla bla. Måske.

%\begin{figure}[H]
%	\centering
%	\includegraphics{tex/Kravspecifikation/Billeder/Aktor-kontekst_diagram_v0_1.pdf}
%	\caption{Aktør-kontekst diagram}
%\end{figure}

%\begin{figure}[H]
%	\centering
%	\includegraphics{tex/Kravspecifikation/Billeder/Usecase_diagram_v1_0.pdf}
%	\caption{Use case diagram}
%\end{figure}

\section{Aktørbeskrivelse}
I det følgende afsnit beskrives systemets aktører. Ved hver aktør angives typen, samt en kort beskrivelse af aktørens funktion og/eller hvordan de påvirker systemet.

\begin{framed}
	\subsection{Aktør: Bruger}
	\subsubsection*{Type:}
		Primær
	
	\subsubsection*{Beskrivelse:}
		Brugeren interagerer med systemet.
		
		Han kan indstille den ønskede load type.
\end{framed}

\begin{framed}
	\subsection{Aktør: Thermal knife}
	\subsubsection*{Type:}
	Sekundær
	
	\subsubsection*{Beskrivelse:}
	Thermal knife er en load type
\end{framed}

\begin{framed}
	\subsection{Aktør: Pyro load}
	\subsubsection*{Type:}
	Sekundær
	
	\subsubsection*{Beskrivelse:}
	Pyro er en load type
\end{framed}

\clearpage


\section{Fully dressed use cases}

\begin{framed}
	\subsection{Use case 1 - Start bil}
	\subsubsection*{Mål:}
		Initiere bilen så den er klar til kørsel og er klar til at modtage input
		
	\subsubsection*{Initiering:}
		Brugeren
	
	\subsubsection*{Aktører:}
		Brugeren (primær)
	
	\subsubsection*{Referencer:}
		Ingen
	
	\subsubsection*{Samtidige forekomster:}
		En
	
	\subsubsection*{Forudsætning:}
		Bilen er slukket og der er forbindelse fra interface til bil
	
	\subsubsection*{Resultat:}
		Bilens sensorer er tændt, motorer er klar, bilen holder stille
	
	\subsubsection*{Hovedscenarie:}
		\begin{enumerate}
			\item Brugeren vælger via interface ''Start bil''
			\item Bilen monitorerer sensorinputs og rapporterer status 
			\item Bilen udfører motortjek ved at køre bilen lidt frem og derefter tilbage
			\item Bilen rapporterer status
			\item Bilen tænder for- og baglys, blinker med blinklys hvis status er OK 
			\begin{description}
					\item[Extension 1:] Status ikke OK
			\end{description}
			\item Bilen afventer brugerinput
		\end{enumerate}
	
	\subsubsection*{Extensions:}
	\textbf{Extension 1:} Status ikke OK	% Fix layout
		\begin{enumerate}
			\item Bilen rapporterer fejl og forsøger at angive hvilken sensor og/eller motor der fejler
		\end{enumerate}
	
\end{framed}
	

% Skabelon
%\begin{framed}
%\subsubsection{Mål:}
%
%\subsubsection{Initiering:}
%
%\subsubsection{Aktører:}
%
%\subsubsection{Referencer:}
%
%\subsubsection{Samtidige forekomster:}
%
%\subsubsection{Forudsætning:}
%
%\subsubsection{Resultat:}
%
%\subsubsection{Hovedscenarie:}
%
%\subsubsection{Extension:}

\section{Ikke-funktionelle krav}
I dette afsnit beskrives de ikke-funktionelle krav. Her opstilles f.eks. krav om præcision, brugervenlighed samt produktets dimensioner.
\begin{itemize}
			\item Inputspændingen skal være mellem 26-100V
			\item Der må maksimalt trækkes en peak-strøm fra inputkilden på 150\% af inputstrømmen
			\item Skal opretholde en outputspænding på op til 21V ved 2,5A
			\item Der må maksimalt være en ripple-spænding på 50mV pk-pk ved fundamental ripple frekvens, og switching spikes på 100mV pk-pk
			\item Skal kunne omsætte op til 75W
			\item Skal operere med et tab på maksimalt 5W %%FIXME
			\item Skal implementeres i et volumen mindre end 17x75x100mm på forsiden af PCB, samt 3x75x100mm på bagsiden PCB'et
			\item Skal kunne operere med en omgivelsestemperatur mellem -35\degreeCelsius  og 65\degreeCelsius
			\item Skal have stabil regulering med 10dB gain og 50 graders fasemargin ved:
				\begin{description}
					\item 21V/2A ved høj og lav indgangsspænding
					\item 5A/2\ohm ved høj og lav indgangsspænding
				\end{description}
			\item Reguleringen skal have en risetime på maksimalt 0,5ms uden overshoot
					
\end{itemize}

	
	\chapter{Accepttest}


%% Skabelon
%\begin{table}[H] 			
%	\centering
%	\begin{tabularx}{\textwidth}{|c|X|X|X|X|}
%		\hline
%		\bfseries Use case under test & \multicolumn{4}{|c|}{< Use case >} \\ \hline
%		\bfseries Scenarie & \multicolumn{4}{| c |}{< Scenarie >} \\ \hline
%		\bfseries Prækondition &  \multicolumn{4}{|c|}{< Prækondition >} \\  \hline
%		\bfseries Step  & \bfseries Handling &  \bfseries Forventet & \bfseries Faktisk & \bfseries Vurdering \\ \hline 
%		\end{tabularx}
%		\caption{< Caption >}
%\end{table}


\section{Test af ikke-funktionelle krav}

\begin{tabularx}{\textwidth}{|X|X|X|X|X|}
	\hline
	\textbf{Krav} & \textbf{Test} & \textbf{Forventet resultat} & \textbf{Resultat} & \textbf{Vurdering} \\ \hline
	Converteren skal kunne operere med en inputspænding mellem 26-50V & Indgangs-spændingen måles med et voltmeter med en load på $8.4\ohm$ Der indsættes voltmeter og amperemeter på udgangen & Indgangs-spændingen er mellem 26-50V og outputspænding ligger på 21V med en strøm på 2.5A & & \\ \hline
	Converteren skal opretholde en outputspænding på 21V $\pm$2\% ved 2,5A $\pm$5\% & Der indsættes en load på 8.4\ohm\ og udgangs-strøm og -spænding måles med oscilloskop & Spændingen ligger på 21V $\pm$2\% og strømmen på 2,5A $\pm$5\% && \\ \hline
	Converteren skal opretholde en outputstrøm op til 5A $\pm$5\% ved 15V $\pm$2\% & Der indsættes en load på 3\ohm\ og udgangs-strøm og -spænding måles med oscilloskop & Spændingen ligger på 15V $\pm$2\% og strømmen på 3A $\pm$5\% && \\ \hline
	Converteren må maksimalt have en output ripple-spænding på 50mV pk-pk & Der indsættes en load på 8.4\ohm\ og pk-pk ripple spænding aflæses med oscilloskop over udgangsloaden & Ripple-spændingen på udgangen er under 50mV pk-pk && \\ \hline
\end{tabularx}



\begin{tabularx}{\textwidth}{|X|X|X|X|X|}
	\hline
	\textbf{Krav} & \textbf{Test} & \textbf{Forventet resultat} & \textbf{Resultat} & \textbf{Vurdering} \\ \hline
	Converteren må maksimalt have switching spikes på 100mV pk-pk & Der indsættes en load på 8.4\ohm\ og pk-pk switching spikes aflæses med oscilloskop over udgangsloaden & Switching spikes aflæses til maksimum 100mV pk-pk && \\ \hline
	Converteren skal kunne omsætte op til 75W & Der indsættes en load på 3\ohm\ og der måles på oscilloskopet om der holdes en spænding på 15V $\pm$2\% samt en strøm på 5A $\pm$5\% & Der måles en spænding på 15V $\pm$2\% samt en strøm på 5A $\pm$5\% hvilket giver 75W && \\ \hline
	Converteren skal operere med et tab på maksimalt 5W & Der indsættes en load på 8.4\ohm\ Indgangs-spænding og strøm måles og omregnes til effekt. Det samme gøres for udgangs-spænding og -strøm. & De 2 effekter trukket fra hinanden giver maksimalt 5W && \\ \hline 
	Converteren skal implementeres i et volumen mindre end 17x75x100mm på forsiden af PCB'et, samt 3x75x100mm på bagsiden af PCB'et & Med målebånd måles dimensionerne af PCB'et først på forsiden og derefter på bagsiden. & Dimensionerne overskrider ikke 17x75x100mm på forsiden af PCB'et og 3x75x100mm på bagsiden af PCB'et && \\ \hline

\end{tabularx}



\begin{tabularx}{\textwidth}{|X|X|X|X|X|}
	\hline
	\textbf{Krav} & \textbf{Test} & \textbf{Forventet resultat} & \textbf{Resultat} & \textbf{Vurdering} \\ \hline
	Converteren skal kunne operere med en omgivelsestemperatur mellem -35\degreeCelsius\ og 65\degreeCelsius\ & Der indsættes en load på 3\ohm\ og der måles på oscilloskopet om der holdes en spænding på 15V $\pm$2\% samt en strøm på 5A $\pm$5\%. Først testes ved -35\degreeCelsius\ og derefter ved 65\degreeCelsius\ & Der måles en spænding på 15V $\pm$2\% samt en strøm på 5A $\pm$5\% hvilket giver 75W ved begge temperature && \\ \hline
	Converteren skal have stabil regulering med minimum 10dB gain og 50 graders fasemargin ved 21V/2,5A ved en indgangsspænding på 26V & Indgangs-spændingen indstilles til 26V og vha. en network analyser genereres et bodeplot ved at måle over loaden.& På bode plottet ses en stabil regulering med 10dB gain og 50 graders fase margin for 26V && \\ \hline
	Converteren skal have stabil regulering med minimum 10dB gain og 50 graders fasemargin ved 21V/2,5A ved en indgangsspænding på 50V & Indgangs-spændingen indstilles til 50V og vha. en network analyser genereres et bode plot ved at måle over loaden. & På bode plottet ses en stabil regulering med 10dB gain og 50 graders fase margin for 50V && \\ \hline
	Converteren skal have stabil regulering med minimum 10dB gain og 50 graders fasemargin ved 5A/3\ohm ved en indgangsspænding på 26V & Indgangs-spændingen indstilles til 26V og vha. en network analyser genereres et bodeplot ved at måle over loaden.& På bode plottet ses en stabil regulering med 10dB gain og 50 graders fase margin for 26V && \\ \hline		
\end{tabularx}


\begin{tabularx}{\textwidth}{|X|X|X|X|X|}
	\hline
	\textbf{Krav} & \textbf{Test} & \textbf{Forventet resultat} & \textbf{Resultat} & \textbf{Vurdering} \\ \hline
	Converteren skal have stabil regulering med minimum 10dB gain og 50 graders fasemargin ved 5A/3\ohm ved en indgangsspænding på 50V & Indgangs-spændingen indstilles til 50V og vha. en network analyser genereres et bodeplot ved at måle over loaden.& På bode plottet ses en stabil regulering med 10dB gain og 50 graders fase margin for 50V && \\ \hline
	Reguleringen skal have en risetime på maksimalt 0,5ms & Ved en load på 8.4\ohm, udgangsstrøm på 2.5A $\pm$5\% og udgangs-spænding på 21V $\pm$2\% måles risetime med et oscilloskop på udgangen ved et step på indgangen & Der måles en risetime på maksimalt 0,5ms && \\ \hline
	Reguleringen skal have et overshoot på maksimalt 5\% & Ved en load på 8.4\ohm, udgangsstrøm på 2.5A $\pm$5\% og udgangs-spænding på 21V $\pm$2\% måles overshoot med et oscilloskop på udgangen ved et step på indgangen & Der måles et overshoot på maksimalt 5\% && \\ \hline
\end{tabularx}
	
	\chapter{Systemarkitektur}

 
Følgende afsnit indeholder SysML BDD og IBD. BDD'et bruges til at give overblik over  systemets hardware blokke, samt hvilke inputs og outputs hver blok indeholder. IBD'et viser forbindelserne mellem hardware blokkene, samt hvilken vej kommunikationen foregår..


\section{Block Definitions Diagram}
\noindent Figur~\ref{fig: BDD} viser et Block Definitions Diagram (BDD) over systemet. Det er med for, at give det første overblik over systemet -- altså hvad systemet består af.

Systemet består af fire hardware blokke -- et Input filter, et Power-modul, en PWM-forsyning, samt et PWM-modul.
Input filteret bruges til at filtrer støj der kommer fra inputkilden, og sikrer en stabil inputspænding. Derudover skal det også filtrere støjsignaler der kan løbe tilbage til kilden.
Power-modulet består af selve convertertrinet. Det er i denne blok inputspændingen bliver konverteret om til den korrekte udgangsspænding.
PWM-forsyningen står for at forsyne PWM-modulet. Under opstart vil blokken regulere converterens inputspænding ned til den korrekte spænding på 12V. Mens converterens output vil bruges når outputspændingen er tilstrækkelig.
PWM-modulet står for selve reguleringen af converterens output. Dette sker ved at overvåge både outputspændingen, samt peak-strøm i power-modulet, og tilpasse PWM-signalets duty-cycle herefter.   

\begin{figure}[H]
	\centering
	\includegraphics[width=1.00\textwidth]{tex/systemarkitektur/billeder/BDD.pdf}
	\caption{BDD}
	\label{fig: BDD}
\end{figure}

\section{Internal Block Diagram}
\noindent Figur~\ref{fig: IBD} viser et Internal Block Diagram (IBD) over systemet. Dette er skridtet efter BDD'et, og viser hvordan systemets blokke er forbundet.


\begin{figure}[H]
	\centering
	\includegraphics[width=1.00\textwidth]{tex/systemarkitektur/billeder/IBD.pdf}
	\caption{IBD}
	\label{fig: IBD}
\end{figure}


\subsection{Signalbeskrivelse}
\noindent Tabel~\ref*{tabel: Signalbeskrivelse} viser en signalbeskrivelse for systemet. Tabellen indeholder signalets type, navn, og en beskrivelse af signalet.

\begin{table}[htbp]
	\centering
	\begin{tabular}{|l|l|l|}
		\hline
		\textbf{Signal type} 	&\textbf{Navn}		&\textbf{Beskrivelse} \\\hline
		26-50V			&Vin		&Ufiltreret inputspænding på 26-50V\\\hline
		26-50V			&Vd			&filtreret inputspænding på 26-50V\\\hline
		26-50V			&VdPWM			&Input til convertering af PWM-controllerens VDD, under opstart\\\hline
		15-21V			&PowerPWM		&Input til convertering af PWM-controllerens VDD, efter opstart\\\hline
		12V				&VDD		&12V forsyning til PWM-controller\\\hline
		15-21V			&Power		&Converterens outputspænding\\\hline
		PWM				&Switch		&PWM signal til regulering af outputspænding\\\hline
		analog			&CS			&Analogt signal til monitorering af peak-strøm   \\\hline	
		analog			&FB			&Analogt signal til monitorering af outputspænding\\\hline
		Vout			&analog		&0-5V signal, som sætter ønsket udgangsspænding\\\hline
		Iout			&analog		&0-5V signal, som sætter ønsket udgangsstrøm\\\hline
		
	\end{tabular}
	\caption{Signalbeskrivelse}
	\label{tabel: Signalbeskrivelse}
\end{table}


	
	% Dokumentation af converter topologi

\chapter{Første Iteration}
I dette afsnit beskrives den indledende og første iteration af designfasen. Den indebærer valg af converter topologi, samt simulering af en ideel converter.

\section{Switch Mode Power-Supply}
I dette projekt vælges der at tage udgangspunkt i Switch Mode Power-Supply (SMPS). Da der er stillet et krav om et maksimalt tab på 5W, betyder det, ved en maksimal udgangseffekt på 75W, at converteren skal have en effektivitet på:
\begin{equation}
	\eta = \frac{75W}{75W + 5W} \cdot 100 = 93.75
\end{equation}

En lineær converter vil ikke kunne opnå så stor en effektivitet. Denne effektivitet kan til gengæld tilnærmes ved brug af en SMPS. Det vælges at tage udgangspunkt i en flyback converter, da denne er ideel til effekter under 100W. 

%TODO find kilder der bekræfter disse påstande.

\section{Flyback Converter}
Flyback converteren, er en transformator baseret topologi. Man deler converteren op i to dele: Primær- og sekundærsiden. Primærsiden består af primærviklingen af transformatoren og en transistor, hvor transistoren fungerer som en switch. Sekundærsiden består af sekundærviklingen, en diode, en udgangskondesator og belastningen. Dette er vist på figur~\ref{fig:flyback_ideal}. En af fordelene ved at bruge flyback converteren er at der kan opnås galvanisk adskillelse mellem primær- og sekundærsiden af transformatoren. Derudover bruges der relativt få komponenter.

\begin{figure}[H]
	\center
	\includegraphics[max width=0.7\linewidth]{/tex/smps/billeder/flyback_ideal.PNG}
	\caption{Ideelt diagram af flyback converteren
	\cite{SMPS-topologies}}
	\label{fig:flyback_ideal}
\end{figure} 

Flyback converteren bruges til at konvertere en indgangsspænding, ned til en mindre udgangsspænding. Dette gøres ved at styre transistoren med et PWM-signal, med en variabel duty-cycle. Når den er ON, vil der være en positiv spænding ved prik-enden af viklingen ift. den anden ende. Ud fra formlen $V=L\cdot \frac{di}{dt}$ kan det ses, at når der ligger en spænding over viklingen, vil strømmen i transformatoren stige lineært, over den tid transistoren er ON. Når transistoren går OFF, vil den magnetiske strøm i transformatoren inducere en spænding over sekundærviklingen. Når denne spænding bliver lig udgangsspændingen, vil dioden begynde at lede den strøm, der er oplagret i transformatoren. Da spændingen over sekundærviklingen er positiv ved prikken, og dermed modsat af primærviklingen, vil strømmen falde lineært ud fra samme forhold, som nævnt tidligere. Dette vil over tid skabe en trekantet kurveform af den samlede strøm i transformatoren. Et eksempel på dette kan ses på figur \ref{fig:CCM_transformer_current}. Da strømmen i hver vikling er diskontinuert, vil det give anledning til større peak-strømme. Det er maksimalt $50\percent$ af tiden der løber en strøm i viklingen, hvilket betyder en større strøm for at opretholde den samme middelstrøm.

Flyback converteren kan overordnet drives på to forskellige måder, Continuous Conduction Mode (CCM) og Discontinuous Conduction Mode (DCM). Disse to måder har forskellige fordele og ulemper, som skal tages højde for inden der vælges hvordan converteren skal drives. 

\subsection{Continuous Conduction Mode}
Forkellen ved CCM og DCM er, hvordan strømmen løber i transformatoren. Ved CCM vil der altid løbe en strøm i transformatoren, som der også ligger i navnet. Dog vil strømmene individuelt i viklingerne være diskontinuerte. Strømmen er skitseret på figur \ref{fig:CCM_transformer_current}. Skal man have den samlede strøm i transformatoren, skal de to kurver for primær- og sekundærviklingen samles. Dette er fordi der kun løber en strøm i primærviklingen når transistoren er ON, og en strøm i sekundærviklingen når transistoren er OFF. 

\begin{figure}[H]
	\center
	\includegraphics[max width=0.7\linewidth]{/tex/smps/billeder/CCM_transformer_current.PNG}
	\caption{CCM transformator strømme
	\cite{SMPS-topologies}}
	\label{fig:CCM_transformer_current}
\end{figure}

\noindent En af fordelene ved CCM er, at strømmen i transformatoren ikke når at aflade helt, inden transistoren går ON igen. Dette giver lavere ripple-strømme, og dermed også peak-strømme, hvilket giver anledning til et mindre effekttab. På grund af den mindre ripple-strøm i transformatoren, opnås der også en mindre ripplespænding på udgangen. hvilket sætter et mindre krav til udgangskondensatoren. 

\subsection{Discontinuous Conduction Mode}
Den anden måde at drive converteren på er DCM. Ved denne metode vil der være en død tid i hver periode, hvor der ikke løber en strøm i transformatoren. Dette betyder at transformatoren når at aflade helt, inden switch-perioden er ovre. Til forskel fra CCM, vil dette give nogle trekantede strømkurver i transformatoren, som ses på figur %TODO indsæt figur.
På grund af død tiden, vil peak-strømmene blive større, da arealet under kurven skal være det samme som ved DCM. fordelen ved at $di$ bliver større, er at induktansen i viklingerne bliver mindre. 

\section{Ideel transformator}
Der vælges at arbejde videre med en flyback converter i CCM, pga. kravet om et tab på maksimalt 5W. På grund af de store strømme i transformatorens viklinger, vurderes det at tabet i MOSFET og diode, vil være for stort når de leder. 

Det startes med at designe en converter der, ved en input spænding på $26V-100V$, kan opretholde en udgangsspænding på $21V$ og $2.5A$. 

\noindent Ud fra dette beregnes en maksimal og minimal duty-cycle:
\begin{equation} \label{D_maks_CCM}
D_{maks} = \frac{V_{out}}{V_{inmin} + V_{out}} = 0.447
\end{equation}
\begin{equation} \label{D_min_CCM}
D_{min} = \frac{V_{out}}{V_{inmaks} + V_{out}} = 0.174
\end{equation}

\noindent Nu kan de maksimale ripple-, peak- og RMS-strømme i transformatoren estimeres: 
\begin{equation} \label{I_ripple_CCM}
I_{ripple} = 0.6 \cdot \frac{V_{out} \cdot I_{out}}{V_{inmaks} \cdot   D_{min}} = 1.8A
\end{equation}
\begin{equation} \label{I_pk_CCM}
I_{pk} = \frac{V_{out} \cdot I_{out}}{V_{inmin} \cdot D_{maks}} + \frac{I_{ripple}}{2} = 5.4A
\end{equation}
\begin{equation} \label{I_pk_avg_CCM}
I_{pkavg} = \frac{I_{out}}{1-D_{maks}} = 4.5A
\end{equation}

\noindent Nu beregnes RMS-strømmene i både primær- og sekundærviklingerne. 
\begin{equation} \label{I_p_RMS_CCM}
I_{RMSp} = \sqrt{D_{maks} \cdot (I_{pkavg})^2} = 3A
\end{equation}
\begin{equation} \label{I_s_RMS_CCM}
I_{RMSs} = \sqrt{(1-D_{maks}) \cdot (I_{pkavg})^2} = 3.4A
\end{equation}

Induktansen i primærviklingen beregnes ud fra den ønskede ripplestrøm, samt switch-frekvensen. Som udgangspunkt vælges den til $100kHz$. Derudover vælges det, at have et omsætningsforhold lig 1 i transformatoren, hvilket betyder $L_p = L_s$.
\begin{equation} \label{L}
L = \frac{V_{inmin} \cdot D_{min}}{I_{ripple} \cdot f_s} = 95.6\micro H
\end{equation}



\section{Udgangskondensator}
I en flyback converter bruges udgangskondensatoren primært til at mindske ripplespændingen på load'en. Formlen for at beregne minimumskapaciteten er
\begin{equation} \label{udgangskondensator}
C_{out} \geqslant \frac{I_{out} \cdot D_{max}}{V_{ripple} \cdot f_s} \geqslant 223.4 \micro F
\end{equation}

\section{Simulering}
Med udgangspunkt i figur~\ref{fig:flyback_ideal} opsættes en ideel flyback converter i p-spice. Dette er gjort på figur~\ref{fig:ideal_flyback_diagram}. Converteren er sat op med en ideel transformatorkobling, et ideelt switching element, samt en ideel diode, for at få et indblik i flyback converterens virkemåde. 


\begin{figure}[H]
	\center
	\includegraphics[max width=0.7\linewidth]{/tex/smps/billeder/flyback_ideal_diagram.PNG}
	\caption{Ideelt flyback kredsløb}
	\label{fig:ideal_flyback_diagram}
\end{figure}

Der er to scenarier der er relevante at kigge på, ved en indgangsspænding på $26V$, samt ved en indgangsspænding på $100V$. Først kigges der på udgangen af converteren, for at kontrollere udgangsstrømmen og -spændingen. På figur~\ref{fig:26V_ideal_output} ses både outputstrømmen (rød) og outputspændingen (grøn), med en inputspænding på 26V. Her ses det at spændingen ligger sig omkring $21V$ og strømmen ligger sig omkring $2.5A$, hvilket var kravet til converteren. Derudover aflæses ripplespændingen til ca. $50mV$, hvilket er overholder kravet for ripplespændingen. 

\begin{figure}[H]
	\center
	\includegraphics[max width=0.7\linewidth]{/tex/smps/billeder/26V_output.PNG}
	\caption{Converter output - ved 26V input}
	\label{fig:26V_ideal_output}
\end{figure}

\noindent På figur~\ref{fig:100V_ideal_output} ses det samme billede, ved 100V inputspænding. Da converterens duty-cycle er faldet, falder ripple-spændingen også. Den aflæses til ca. $20mV$. 
%TODO Forklar mærkelig kurve form.

\begin{figure}[H]
	\center
	\includegraphics[max width=0.7\linewidth]{/tex/smps/billeder/100V_output.PNG}
	\caption{Converter output - ved 100V input}
	\label{fig:100V_ideal_output}
\end{figure}

\noindent I tabel~\ref{tab:result_ideal_converter} ses resultaterne for analyse(A) og simulering(S), af den ideelle converter. Ripple- og peakstrømmene er aflæst ud fra figur~\ref{fig:26V_transformer_current} og \ref{fig:100V_transformer_current}. RMS-strømmene findes ved, at bruge RMS-funktionen i p-spice. Derudover kan det konstateres at converteren operer i CCM, da transformatorstrømmen ikke når at aflade helt. Se figur~\ref{fig:CCM_transformer_current}. 

\begin{table}[H] 			
	\centering
	\begin{tabularx}{\textwidth}{|X|c|c|c|c|c|c|c|c|}
		\hline
		\textbf{Indgangs-spænding} & \multicolumn{2}{|X|}{\textbf{Ripplestrøm}} & \multicolumn{2}{|X|}{\textbf{Peakstrøm}} & \multicolumn{2}{|X|}{\textbf{RMS-strøm i primær}} & \multicolumn{2}{|X|}{\textbf{RMS-strøm i sekundær}} \\ \hline
		& A & S & A & S & A & S & A & S \\ \hline
		$26V$ & $1.2A$ & $1.2A$ & $5.1A$ & $5.1A$ & $3.0A$ & $3.0A$ & $3.4A$ & $3.4A$ \\ \hline 
		$100V$ & $1.8A$ & $1.8A$ & $3.9A$ & $3.9A$ & $1.3A$ & $1.3A$ & $2.8A$ & $2.8A$ \\ \hline
	\end{tabularx}
	\caption{Resultater for analyse og simulering af ideel flyback converter}
	\label{tab:result_ideal_converter}
\end{table}

\begin{figure}[H]
	\center
	\includegraphics[max width=0.7\linewidth]{/tex/smps/billeder/26V_transformer_current.PNG}
	\caption{Transformator strømme - ved 26V input}
	\label{fig:26V_transformer_current}
\end{figure}

\begin{figure}[H]
	\center
	\includegraphics[max width=0.7\linewidth]{/tex/smps/billeder/100V_transformer_current.PNG}
	\caption{Transformator strømme - ved 100V input}
	\label{fig:100V_transformer_current}
\end{figure}




	
	%\include{tex/design/design}

	%\include{tex/Done_acceptest/Done_acceptest}
	
	%\include{<sti/filnavn.tex>}

	
	%%% Foreløbig disposition %%%
	% Introduktion
	%% Projektformulering/Motivation?
	%% Systembeskrivelse
	%% Systembetragtning
	% Kravspecifikation
	% Systemarkitektur
	%% HW
	%% SW
	% Accepttest
	
	\printbibliography[title={Litteraturliste}]
	
\end{titlingpage}
\end{document}