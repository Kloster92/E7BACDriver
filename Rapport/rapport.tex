%%%% Semesterprojekt rapport gruppe 2 %%%%
%%%% Elektronik Q1/Q2 2016 %%%%

% Kan anvendes til journaler eller afleveringer
\documentclass[11pt, a4paper, twoside, openany]{memoir}

\usepackage[utf8]{inputenc}		% Dansk input encoding (tegn)
\usepackage[english, danish]{babel}		% Danske formuleringer / orddeling
\usepackage[T1]{fontenc}		% Output-indkodning af tegnsaet (T1)


%%% Memoir indstillinger
%% Afstand mellem afsnit og videre
%% NIX PILLE - medmindre strengt nødvendigt
\setaftersubsubsecskip{6pt}	
\setbeforesubsubsecskip{ 6pt}
%\setaftersubsecskip{6pt}
%\setbeforesubsecskip{-\baselineskip}
%\setaftersecskip{6pt}
%\setbeforesecskip{-\baselineskip}
%\setaftersecskip{1ex}


\raggedbottom



\chapterstyle{section}

% ¤¤ Marginer ¤¤ %
\setlrmarginsandblock{3.5cm}{2.5cm}{*}		% \setlrmarginsandblock{Indbinding}{Kant}{Ratio}
\setulmarginsandblock{3.0cm}{2.5cm}{*}		% \setulmarginsandblock{Top}{Bund}{Ratio}
\checkandfixthelayout

%%% Font valg %%%
\usepackage{mathpazo}	%Palatinofont - matematikformler
\usepackage{eulervm}		%Palatinofont

%%% FIGURER OG TABELLER %%%
\usepackage{graphicx} 						% Haandtering af eksterne billeder (JPG, PNG, PDF)

\usepackage[export]{adjustbox}

\usepackage{subfig}

\usepackage{multirow}                		% Fletning af raekker og kolonner (\multicolumn og \multirow)
\usepackage{colortbl} 						% Farver i tabeller (fx \columncolor, \rowcolor og \cellcolor)
\usepackage[dvipsnames]{xcolor}				% Definer farver med \definecolor. Se mere: http://en.wikibooks.org/wiki/LaTeX/Colors
%\usepackage{flafter}						% Soerger for at floats ikke optraeder i teksten foer deres erence
\usepackage{float}							% Muliggoer eksakt placering af floats, f.eks. \begin{figure}[H]
\usepackage{multicol}         	        	% Muliggoer tekst i spalter
%\usepackage{rotating}						% Rotation af tekst med \begin{sideways}...\end{sideways}
\usepackage{booktabs}
\usepackage{bigstrut}	% Excel2latex måskeh
\usepackage{tabularx}

%%% ¤¤ Matematik mm. %%%
\usepackage{amsmath,amssymb,stmaryrd} 		% Avancerede matematik-udvidelser
\usepackage{mathtools}						% Andre matematik- og tegnudvidelser
\usepackage{textcomp}                 		% Symbol-udvidelser (f.eks. promille-tegn med \textperthousand )
\usepackage{siunitx}						% Flot og konsistent praesentation af tal og enheder med \si{enhed} og \SI{tal}{enhed}
\sisetup{output-decimal-marker = {,}}		% Opsaetning af \SI (DE for komma som decimalseparator) 
\sisetup{exponent-product=\cdot, output-product=\cdot}	%Eksponent er gange tegn, output produkt er gange tegn
\sisetup{digitsep = none}					%Almindeligt komma - ingen mellemrum aka. til eurokomma




%%% MISC %%%
\usepackage{listings}						% Placer kildekode i dokumentet med \begin{lstlisting}...\end{lstlisting}
\definecolor{bg}{HTML}{F0F0F0}
\lstset{language=C++,
				showstringspaces = false,
				backgroundcolor = \color{bg},
                basicstyle=\ttfamily,
                keywordstyle=\color{blue}\ttfamily,
                stringstyle=\color{red}\ttfamily,
                commentstyle=\color{green}\ttfamily,
                morecomment=[l][\color{magenta}]{\#},
                extendedchars=true,
                numbers=left, numberstyle=\tiny,		% Linjenumre
                columns=flexible,						% Kolonnejustering
                breaklines, breakatwhitespace=true,		% Bryd lange linjer
                literate=%
                {æ}{{\ae}}1
                {å}{{\aa}}1
                {ø}{{\o}}1
                {Æ}{{\AE}}1
                {Å}{{\AA}}1
                {Ø}{{\O}}1
}


\usepackage{lipsum}							% Dummy text \lipsum[..]
\usepackage[shortlabels]{enumitem}			% Muliggoer enkelt konfiguration af lister
\usepackage{pdfpages}						% Goer det muligt at inkludere pdf-dokumenter med kommandoen \includepdf[pages={x-y}]{fil.pdf}	
\pdfoptionpdfminorversion=6					% Muliggoer inkludering af pdf dokumenter, af version 1.6 og hoejere

%	¤¤ Afsnitsformatering ¤¤ %
%\setlength{\parindent}{0mm}           		% Stoerrelse af indryk
\setlength{\parskip}{1.5mm}          		% Afstand mellem afsnit ved brug af double Enter
\linespread{1,1}							% Linie afstand

\usepackage{tikz}


% ¤¤ Visuelle  ¤¤ %
\usepackage[colorlinks]{hyperref}			% Danner klikbare referencer (hyperlinks) i dokumentet.
\hypersetup{colorlinks = true,				% Opsaetning af farvede hyperlinks (interne links, citeringer og URL)
	linkcolor = black,
	citecolor = black,
	urlcolor = black
}
\usepackage{url}

%%% REFERENCER %%%
%\usepackage{xr}
%\externaldocument{../dokumentation/dokumentation.tex}



%%% Referencer / Bibliografi %%%
\usepackage[backend=bibtex, sorting=none, style=numeric]{biblatex}
\bibliography{../referencer.bib}






\usepackage[draft, danish]{fixme}
\fxsetup{layout=footnote}

\graphicspath{{../fig/}{../fig}{fig/}{./}}

\usepackage{titlesec}

\setcounter{secnumdepth}{4}

\titleformat{\paragraph}
{\normalfont\normalsize\bfseries}{\theparagraph}{1em}{}
\titlespacing*{\paragraph}
{0pt}{3.25ex plus 1ex minus .2ex}{1.5ex plus .2ex}


%%%% Opsætning af dokument %%%%
\newcommand{\forfatter}{Gruppe 2}
\newcommand{\fag}{INDSÆT KURSUS HER}
\newcommand{\titel}{Semesterprojekt 4 Rapport}
\date{}

\author{\forfatter}
\title{\titel}


\setlength{\beforechapskip}{10pt}
\setlength{\afterchapskip}{10pt}
\begin{document}
%\maketitle
\begin{titlingpage}
%\thispagestyle{title}
\selectlanguage{danish}
		
		\begin{center}
				{\huge\bfseries Projekt AutoCar}\\
				\vspace{10pt}
				
				{\Huge\bfseries Rapport}\\
				
				\vspace{20pt}
				
				{Diplomingeniør Elektronik}\\
				{\large 4. Semesterprojekt forår 2016}\\
				
				\vspace{10pt}
				
				Ingeniørhøjskolen Aarhus Universitet\\
				Vejleder: Lars G. Johansen
				\vspace{10pt}
				
				27. maj 2016
				\vspace{10pt}
				\begin{figure}[H]
					\centering
					%\includegraphics[max width=0.9\linewidth]{forside.png}
				\end{figure}
				\vspace{50pt}
				\begin{minipage}{0.25\linewidth}
					\centering
					\hrule
					\vspace{12pt}
					Jonas Baatrup\\
					Studienr. 201405146
				\end{minipage}
				\hspace{10pt}
				\begin{minipage}{0.25\linewidth}
					\centering
					\hrule
					\vspace{12pt}
					Troels Ringbøl Brahe\\
					Studienr. 20095221
				\end{minipage}
				\hspace{10pt}
				\vspace{20pt}
				\begin{minipage}{0.25\linewidth}
					\centering
					\hrule
					\vspace{12pt}
					Nicolai H. Fransen\\
					Studienr. 201404672
				\end{minipage}
				\hspace{10pt}
				\begin{minipage}{0.25\linewidth}
					\centering
					\hrule
					\vspace{12pt}
					Jesper Kloster\\
					Studienr. 201404571
				\end{minipage}
				\hspace{10pt}
				\begin{minipage}{0.25\linewidth}
					\centering
					\hrule
					\vspace{12pt}
					Rasmus Harboe Platz\\
					Studienr. 201408608
				\end{minipage}
				\hspace{10pt}
				\vspace{20pt}
				\begin{minipage}{0.25\linewidth}
					\centering
					\hrule
					\vspace{12pt}
					Nicolai Bonde\\
					Studienr. 201404519
				\end{minipage}
				\begin{minipage}{0.25\linewidth}
					\centering
					\hrule
					\vspace{12pt}
					Emil Jepsen\\
					Studienr. 20092013
				\end{minipage}
		\end{center}
\clearpage
			\noindent\textbf{Ansvarsområder}
Tabellen nedenfor viser de primære ansvarsområder for hver af gruppens medlemmer.
			\begin{table}[H]
				\centering
				\begin{tabular}{| c | c |}
					\hline
					\bfseries Navn & \bfseries Ansvarsområder \\ \hline
					Jonas Baatrup & Motorstyring, Motor-design, Regulering\\ \hline
					Nicolai Bonde &  RPi kode, Bil-design, Forsyning\\ \hline
					Jesper Kloster &  Servo, Vejbanesensorer, Tachometer\\ \hline
					Nicolai Fransen &  Servo, Vejbanesensorer, Tachometer\\ \hline
					Troels Brahe &  Sonar, Regulering, RPi kode\\ \hline
					Emil Jepsen &  Motorstyring. Regulering, SPI\\ \hline
					Rasmus Platz & TCP, GUI  \\ \hline
				\end{tabular}
				\caption*{Tabel over ansvarsfordeling i projektet}
			\end{table}
		\end{titlingpage}
		
		\selectlanguage{danish}
		\begin{abstract}
			Denne rapport beskriver udviklingen af et 4. semester-projekt på IHA. Problemstillingen omhandler design og implementering af en selvkørende bil, AutoCar. Auto\-Car har en DC-motor til fremdrift, en servo-motor til retningsstyring, en sonar-sensor til afstandsbedømmelse, en vejbanesensor i hver side til detektion af vejbanestriber, to batterisensorer til at vise batteriniveauerne, et tachometer til at finde AutoCar's fart, og intern logik, regulering, og kommunikation for at få det hele til at fungere sammen. 
			Brugeren kan via interfacet starte og stoppe AutoCar, få udskrevet status fra sensorne, sætte farten og anmode om overhaling. AutoCar skal selv sørge for at køre inden for vejbanestriberne.
			
			AutoCar er udviklet med en PSoC 4 og en Raspberry Pi 2b, der tilsammen fungerer som kontrolenhed for AutoCar. GUI'et er designet med QT Creator i en Linux-terminal.
			
			Udviklingsprocessen har båret præg af iterative værktøjer som SCRUM og V-modellen, og med fokus på design fra bunden og op. ASE-modellen for projektudvikling er også benyttet i den tidsmæssige planlægning. Brugen af disse modeller har hjulpet i udviklingen af projekt. Projektprocessen er mundet ud i en prototype hvor størstedelen af funktionaliteten er implementeret.
			
		\end{abstract}
		
		\selectlanguage{english} %%FIXME
		\begin{abstract}
			This report describes the devleopment of a 4th semester project at IHA concerning the design and development of a self-driving car, AutoCar.
			AutoCar has a DC-motor for propulsion, a servo for directional control, a sonar sensor used for ranging, a road sensor in each side to detect road markings, two battery sensors to detect and display battery levels, a tachometer to measure the speed of AutoCar, and a lot of internal logic, regulation, and communication to make it all work. 
			The user can use the interface to start and stop AutoCar, print the status from the sensors, set the speed, and a request overtaking. AutoCar is itself responsible for driving within the road markings.
			
			AutoCar is developed with a PSoC 4 and a Raspberry Pi 2b, which collectively function as the control unit. The GUI is designed with QT Creator in a Linux terminal.
			
			The devleopment process has been influence by iterative tools such as SCRUM and the V model, and a strong focus on a bottom-up approach to the design process. The ASE Development Model has been used for chronological planning, and redmine along with SmartGit has been used to keep track of project files and resources. The use of these models and tools has ensured a competently devleloped system.
		\end{abstract}
		
		\selectlanguage{danish}
		
		\clearpage
		
	%\setcounter{tocdepth}{2}
	\tableofcontents
	\clearpage
	
	%% Synopsis:
	
	% Indholdsfortegnelse
	% Ansvarsområder
	% Abstract
	
	% Forord
	% Indledning
	% Opgaveformulering
	% Systembeskrivelse
	% Krav(specifikation)
	% Projektbeskrivelse
	%% Projektgennemførsel
	%% Udviklingsprocesser / Arbejdsmetoder
	%% (Udviklingsværktøjer); nødvendigt?
	%% Specifikation og Analyse
	%% Systembetragtning
	%% Systemarkitektur
	
	% Hardware - design og implementering
	%% Design
	%% (De forskellige moduler)
	
	% Software - design og implementering
	
	% Resultater og diskussion
	% Fremtidigt arbejde (?)
	% Konklusion
	% Ordliste (?)
	% Referencer
	
	%\include{<sti/filnavn>}

\end{document}