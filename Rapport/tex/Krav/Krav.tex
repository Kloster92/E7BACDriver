
\chapter{Krav}
Projektets krav er specificeret, og prioriteret, vha. MoSCoW-metoden\cite{MoSCoW}. Metoden deler kravene til produktet op i fire kategorier - Must, Should, Could og Won't. Her er der prioriteret den grundlæggende funktionalitet, ved at udvikle en funktionsdygtig converter med en stationær udgang. Med dette udgangspunkt er yderligere funktionaliteter blevet nedprioriteret, og placeret i \textit{Should} og i \textit{Could}. 

\begin{itemize}
	\item[\textbf{Must}]
	\begin{itemize}
		\item Have et funktionsdygtigt power-modul
		\item Have stabil regulering
		\item Underbygges med en P-Spice model
		
	\end{itemize}
	\item[\textbf{Should}]
	\begin{itemize}
		\item Have et termisk design, kompatibelt med vakuum
		\item Have mulighed for brug af to forskellige foruddefinerede load typer
		\item Have overstrømsbeskyttelse på udgangen
		\item Have overspændingsbeskyttelse på udgangen
		\item Ikke påvirke andre moduler ved fejl
		
	\end{itemize}
	\item[\textbf{Could}] 
	\begin{itemize}
		\item Konstrueres med EEE komponenter
		\item Overholde et specifikt temperaturinterval
		\item Implementeres på et standard Terma modul 
		
	\end{itemize}
	\item[\textbf{Won't}]
	\begin{itemize}
		\item Have mulighed for brug til mere end to forskellige typer loads
		\item Have feedback til brugeren når valgt load er aktiveret
		\item Have galvanisk adskillelse
		
	\end{itemize}
\end{itemize}

\section{Kravspecifikation}
Kravene til produktet er opstillet som ikke-funktionelle krav. Det er krav, der fortæller noget om kvaliteten af converteren. Det indebærer krav til indgangsspændingen, præcision af udgangen og det maksimale effekttab i converteren. Der er også stillet krav til operation ved to forskellige belastninger. Samtlige ikke-funktionelle krav er beskrevet i dokumentationen, afsnit 1.3.

