\subsection{MOSFET, diode og udgangskondensator}
I dette afsnit beskrives komponentvalg, design og test af MOSFET, diode og udgangskondensatoren.

\subsubsection{MOSFET}
Ved MOSFET'en stilles der krav til den maksimale drain-source spænding, den kan holde til. Ideelt set er det den maksimale indgangsspænding plus udgangsspændingen. Udgangsspændingen, da det er den spænding, der reflekteres tilbage til primærviklingen fra sekundærviklingen, når MOSFET'en skifter. Som en sikkerhedsmargin er der indlagt $30\percent$ oveni spændingen. På den måde tages der højde for peakspændinger ved switching. Det er peaks, der ikke kan undgås, og kommer som en kombination af spredningsselvinduktionen og kapaciteterne i MOSFET'en. Der er og en kapacitet i koblingen for transformatoren, men typisk er MOSFET'ens dominerende. 


Udover spændingskravet er der krav til at MOSFET'en skal kunne holde til den udregnede RMS- og peakstrøm. Med kravene in mente er valget faldet på MOSFET'en IRFB23N15\cite{IRFB23N15} til 2. iteration. Denne kommer med en $R_{ds(on)}$ modstand på ca. $113\milli \ohm$ ved $50\degreeCelsius$

Switch-tiden for MOSFET'en afhænger af strømmen der løber i den. For en specifik MOSFET er især Miller kapaciteten vigtig at tage højde for. Den påvirker opladningen af gate-drain og bestemmer hvor hurtigt MOSFET'en kan skifte fra OFF til ON. Grundet dette kan en MOSFET ikke skifte momentant og derved opstås switchtab. Switch-tiden er et tradeoff mellem at opnå et lille tab og ikke få for høje peakspændinger. Til 2. iteration er der valgt en forholdsvis langsom switchtid på $138.7ns$, hvilket er gjort med en gate modstand på $51.1\ohm$

\fxnote{Bør der forklares mere om hvordan gate modstanden findes?} 

\subsubsection{Diode}
Dioden kan hurtigt give anledning til store tab. Derfor er der ved udvælgelse af den fokuseret på at finde en diode med et lille spændingsfald. Derudover en Schottky diode, så der ikke kommer et switch-tab grundet reverse recovery. Diodens breakdown voltage skal være høj nok til at holde til spændingen der ligger over den, i MOSFET'ens ON periode. Det indebærer den maksimale indgansspænding plus udgangsspændingen. Igen er der designet efter en sikkerhedsmargin på $30\percent$, for at tage højde for spændingspeaks. Dioden skal samtidig kunne holde til den udregnede RMS- og peakstrøm der vil løbe i den. Her er Schottky dioden NTSV30120CT\cite{NTSV30120} valgt. Spændingsfaldet er aflæst til $0.45V$ ved $125\degreeCelsius$ og $2.5A$.  

\subsubsection{Udgangskondensator}
Som udgangskondensatorer er der valgt 4 parallelle film kondensatorer af typen PET B32526~\cite{Kondensator}. for at opnå den ønskede kapacitet. Film kondensatorer er valgt, da de typisk kommer med præcise kapaciteter og en lav ESR modstand.

Både ESR modstand og ESL induktans er testet ved hjælp af en impedansmåler. Her er der testet med korte ledninger samt en 4-wire teknik, for at undgå måleparasitter.

For yderligere forklaring af MOSFET, diode og udgangskondensator henvises til dokumentation afsnittene 5.2, 5.3 og 5.4, hvor dette er uddybet.