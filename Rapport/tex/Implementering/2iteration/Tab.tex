\subsection{Tab} \label{tab}
Dette afsnit omhandler de overvejelser, der er gjort omkring tab i 2. iteration. Her omtales hvilke bidrag, der er taget højde for, til udberegninger af tab for de enkelte komponenter. Udregningerne og yderligere forklaring af disse, kan findes dokumentationen afsnit 5.7.

\subsubsection{Transformator}
\noindent Tabet i transformatoren er set som to dele. Et kernetab og et kobbertab. Kernetabet afhænger af  kernematerialet, selvinduktionen og strømmen i viklingerne. Disse bruges til at udregne fluxen i kernen. Med denne værdi og databladskurven for det specifikke tab, er kernetabet blevet estimeret. 

Kobbertabet kommer af modstanden i kobbertrådene, som er viklet om kernen. Dette indebærer både bidrag fra en DC modstand og en AC modstand. DC modstanden er udregnet ud fra længden og tykkelsen af kobbertrådene. AC modstanden opstår på grund af magnetfeltet kobbertrådene ligger i. AC modstandens bidrag til kobbertabet er der ikke taget højde for i dette projekt.  

\subsubsection{MOSFET}
\noindent MOSFET'ens tab kan ligeledes deles op i to centrale bidrag - conduction tab og switch-tab. Conduction tabet kommer af RMS strømmen der løber i MOSFET'ens ON modstand. 

Switch-tabet kommer som konsekvens af effekttrekanterne, der opstår, imellem MOSFET'ens ON og OFF perioder. Effekttrekanterne er i dette projekt estimeret ved at udregne dem som arealet af to lige store trekanter. Højden på trekanterne er peakaverage strømmen ganget med den maksimale spænding der vil ligge over MOSFET'en. Længden af trekanten fås af den samlede switch-tid i forhold til den samlede switch periode. 

\subsubsection{Diode}
Tabet i dioden er udregnet ved, at kigge på spændingsfaldet over dioden ganget udgangsstrømmen. Som nævnt benyttes en schottky diode, og der har derfor ikke været behov for betragtninger af switch-tabet i dioden. 
  

\subsubsection{Kondensator}
Med en kendt ESR modstand for kondensatoren, har det været muligt at beregne tabet i denne. Da modstanden er så lille, er tabet dog uden betydning for det samlede tab.

\subsubsection{Current-sense tab}
Tabet i current-sense modstanden er udregnet ved modstandsværdien ganget med RMS strømmen i anden. Strømmen i igennem modstanden er den samme som løber i den primære vikling. 