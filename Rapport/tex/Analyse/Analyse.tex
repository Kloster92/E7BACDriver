
\chapter{Analyse}
I dette afsnit beskrives de betragtninger og valg, der er foretaget i løbet af produktudviklingen. Der er vurderet på hvilke elementer, der er nødvendige for realisering af produktet. For hvert af elementerne, er der foretaget en vurdering af hvilke muligheder og alternativer de har. Ved teknologierne, som er undersøgt, vurderes fordele og ulemper. Ud fra vurderingerne er valget af de brugte teknologier foretaget. 

\section{Converter topologi}
Converter topologien er en essentiel del af converterens endelige funktionalitet. Derfor er der undersøgt fordele og ulemper ved flere forskellige typologier. Converteren skal kunne konvertere indgangsspændingen til en mindre udgangsspænding. Derfor er der udelukkende undersøgt topologier med denne funktionalitet. De undersøgte topologier er buck converteren, push-pull converteren og flyback converteren. Disse convertere er yderligere beskrevet i dokumentationens afsnit 4.

\subsection{Buck converter}
Buck converteren er en af de mere simple converter topologier, da den består af relativt få komponenter. Udover det er en af fordelene ved en buck converter, at der altid vil løbe en strøm i spolen. Det vil genere en minimal ripple-spænding på udganen, og derfor også et lille tab i udgangsfilteret. 

En stor ulempe ved buck converteren er, at switch-komponenten er placeret i den positive forsyningslinje. Det giver komplikationer ift. at drive for eksempel en MOSFET. Det vil kræve flere komponenter, og derfor også et mere kompliceret kredsløb. Af denne grund er buck converteren blevet fravalgt\cite{buck-converter}.

\subsection{Push-pull converter}
Push-pull converteren er en mere avanceret converter topologi. Det er en transformator baseret converter, som giver en ekstra mulighed for effektoverførsel. Transformatoren giver desuden muligheden for sikre galvanisk adskillelse mellem indgangen og udgangen. Push-pull converteren består af to switch-grene på både primær- og sekundærsiden af transformatoren. Det vil sige, den har to primære- og to sekundære viklinger, samt to switch-elementer og to dioder. Den kræver derfor flere komponenter end andre topologier. 

En ulempe ved, at have to switch-grene er uligheder i dem. Tolerancer i komponenterne vil få kernematerialet til at drive mod mætning, hvis der ikke tages højde for det i reguleringen. Derfor er det utilstrækkeligt, udelukkende at regulere efter spændingen i en push-pull converter. En anden konsekvens ved brug af to switch-grene, er død-tid i transformatoren. Dette skal indføres for at sikre, der ikke sker en kortslutning af primærviklingen, hvis begge switch-elementer er ON på samme tid. Desuden skal den valgte PWM-controller understøtte switching af to switch-elementer. 

En fordel ved push-pull converteren er en bedre udnyttelse af kernematerialet, da det vil blive drevet i både 1. og 3. kvadrant af hysteressekurven. 

På grund af kompleksiteten, og mængden af komponenter, er push-pull converteren blevet fravalgt. 

\subsection{Flyback converter}
Flyback converteren er en videreudvikling af buck converteren. Det er også en transformator baseret typologi. Transformatoren erstatter spolen i buck converteren og derfor består de af det samme antal komponenter.

En ulempe ved flyback converteren er en diskontinuert strømform i transformatoren, da der ikke løber strøm i primær- og sekundærviklingen på samme tid. Det vil skabe større ripple- og RMS-strømme, og derved også genere et større tab i komponenterne\cite{SMPS-topologies}. 

En flyback converter kan drives på to overordnede måder - CCM og DCM. De to metoder bidrager yderligere med hver deres fordele og ulemper, som også vil blive beskrevet. 

Forskellen på de to metoder ligger i kurveformen for strømmen, der løber i transformatorviklingerne. Ved CCM vil der altid løbe en strøm i enten den ene, eller den anden vikling. Ved DCM vil strømmen have en dødtid i løbet af en switch-periode. For at DCM skal kunne opretholde den samme udgangsstrøm som CCM, vil det betyde, at peak- og RMS-strømmene bliver større. Det giver derved anledning til et større effekttab. 

Fordelen ved at operere i DCM er primært en simplificering af reguleringssløjfen. CCM indeholder et dominerende nulpunkt langt nede i frekvens, der kan gøre systemet ustabilt, hvis der ikke tages højde for det. Nulpunktet begrænser båndbredden, og dermed også systemets responstid. 

Ud fra disse undersøgelser er blev det valgt, at arbejde videre med en flyback converter opererende i CCM. Det vil sikre en converter, hvor det er muligt, at holde effekttabet og kompleksiteten på et acceptabelt niveau. 

\section{Reguleringsmetode}
Der er to overordnede reguleringsmetoder, der bruges til regulering i en DC/DC converter - Spændingsregulering og strømregulering. Ved spændingsregulering vil reguleringen udelukkende ske, på baggrund af et spændingsfeedback til reguleringssløjfen. Ved strømregulering reguleres der både efter strømmen og spændingen på udgangen. Det sikrer en overstrømsbeskyttelse for converteren.

\noindent Derfor vælges det, at implementere en strømregulering i converteren. 

\section{PWM-controller}
PWM-controlleren er en central del af en DC/DC converter. Det er controlleren, der står for, at generere switch-signalet til converterens switch-element. Af denne grund er der mange krav, PWM-controlleren skal leve op til. Den skal først og fremmest understøtte den valgte reguleringsmetode. Desuden skal den kunne generere den maksimale duty-cycle, som vil kunne fremstå i converterens switch-signal. Den skal kunne opretholde den valgte switch-frekvens. Derudover skal den kunne generere et switch-signal med en amplitude der er høj nok til, at drive den valgte MOSFET.

Ud fra kravene, er det valgt at bruge en PWM-controller af typen UCC1801\cite{UCC1801}. Controlleren lever op til de førnævnte krav, og desuden understøtter den peak-current regulering. Den har indbyggede reguleringssløjfer, således nødvendigheden for eksterne komponenter mindskes. Derudover indeholder den flere sikkerhedsfunktioner ved opstart. Det er samtidig en controller Terma har erfaringer med, og kan omsætte til en EEE-komponent. 

\section{Transformator} \label{Transana}
Størrelsen på transformatoren betyder meget ift. temperaturstigningen i den. Det især den termiske modstand i transformatoren, som er afhængig af størrelsen\cite{epcos-cores}. Derfor er det valgt at bruge en RM8\cite{RM8} kerne, som er den største, der overholder kravene for dimensionerne. Desuden er kernetypen RM blevet anbefalet af Terma, der har haft god erfaring med brug af disse i lignende konfigurationer. Det er valgt at bruge kernematerialet 3f3\cite{3f3}, da Terma har mere præcise målinger af materialets specifikationer, ift. databladets tolerancer.

