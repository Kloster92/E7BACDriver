

\section{Opgaveformulering}
I projektet skal der udvikles en DC/DC converter, som del af et universelt aktiveringskredsløb.   

\noindent Følgende punkter skal indgå i produktet:
\begin{itemize}
	\item Converteren kan holde en stabil udgangsspænding, ved ændring af indspænding
	\item Converteren indeholder en præcis regulering af udgangen, efter både udgangsspænding og -strøm
	\item Converteren kan programmeres til to forskellige udgangsbelastninger, ved to analoge spændinger
	\item Converteren indeholder stabil regulering af udgangen, ved både laveste og højeste indgangsspænding
	\item Converteren har et termisk design, der er funktionsdygtigt i vakuum 
\end{itemize}

\subsubsection*{Indgangsspænding}
\noindent Termas kunder har ofte forskellige krav til converterens indgangsspænding. Derfor skal converteren designes til operation med stor variation i indgangsspændingen. Det vil fremtidssikre produktet, og sikrer en lav tilpasningsmængde senere hen.  

\subsubsection*{Udgangsspænding}
\noindent Converteren skal indgå i et universelt aktiveringskredsløb. Derfor skal det være muligt at programmere udgangsspændingen, således den tilpasses den ønskede belastningstype. Til det er der stillet to analoge signaler til rådighed. 

\subsubsection*{Overspænding- og overstrømsbeskyttelse}
\noindent Flere af de ønskede belastningstyper består af en glødetråd i en metallisk beholder. Glødetråden vil brænde over, og vil muligvis skabe en kortslutning ud til den metalliske beholder. Derfor skal converteren indeholde en overspændings- og overstrømsbeskyttelse. Det vil samtidig sikre en optimal beskyttelse af converteren imod fejl.

\subsubsection*{Hurtig og præcis regulering}
\noindent Princippet i flere af belastningstyperne er pyroteknik. For at sikre en sikker og pålidelig antændelse af krudtladningen, skal converteren indeholde en hurtig og præcis regulering af udgangen ved aktivering. 

\subsubsection*{Termisk design}
\noindent Varmeafledning er begrænset ved rumfart. For at sikre tilstrækkelig levetid på converteren, skal den kunne operere kontinuerligt i vakuum uden overophedning. Derfor skal converteren have en stor effektivitet af effektoverførsel fra indgangskilde til belastning. 

\fxnote{Oversigt over ansvarsområder}
