

\section{Opgaveformulering}
I projektet skal der udvikles en DC/DC converter, som del af et universelt aktiveringskredsløb.   

\noindent Følgende punkter skal indgå i produktet:
\begin{itemize}
	\item Converteren kan holde en stabil udgangsspænding, ved ændring af indspænding
	\item Converteren indeholder en præcis regulering af udgangen, efter både udgangsspænding og -strøm
	\item Converteren kan programmeres til to forskellige udgangsbelastninger, ved to analoge spændinger
	\item Converteren indeholder stabil regulering af udgangen, ved både laveste og højeste indgangsspænding
	\item Termisk design der er funktionsdygtigt i vakuum 
\end{itemize}

\subsubsection*{Indgangsspænding}
\noindent Terma's kunder har ofte forskellige krav til converterens indgangsspænding. Derfor skal converteren designes til operation med stor variation i indgangsspændingen. Det vil fremtidssikre produktet, sikre en lav tilpasningsmængde ved salg til forskellige kunder.  

\subsubsection*{Udgangsspænding}
\noindent Converteren skal indgå i et universelt aktiveringskredsløb. Derfor skal det være muligt at programmere udgangsspændingen, således den tilpasses den ønskede load type. Til dette er der stillet to analoge signaler tilrådighed. 

\subsubsection*{Overspænding- og overstrømsbeskyttelse}
\noindent Flere af de ønskede load typer består af en glødetråd i en metallisk beholder. Glødetråden vil brænde over, og derfor muligvis skabe en kortslutning ud til den metalliske beholder. Derfor skal converteren indeholde en overspænding- og overstrømsbeskyttelse. Det vil samtidig sikre en optimal beskyttelse af converteren mod fejl.

\subsubsection*{Hurtig og præcis regulering}
\noindent Princippet i flere af load typerne er pyroteknik. For at sikre en sikker og pålidelig antændelse af krudtladningen, skal converteren indeholde en hurtig og præcis regulering af udgangen ved aktivering. 

\subsubsection*{Termisk design}
\noindent Ved rumfart er varmeafledning begrænset. For at sikre tilstrækkelig levetid på converteren, skal den kunne operere kontinuerligt i vakuum uden overophedning. Derfor skal converteren have en stor effektivitet af effektoverførsel fra indgangskilde til load. 

\fxnote{Oversigt over ansvarsområder}
