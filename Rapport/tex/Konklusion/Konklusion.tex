
\chapter{Konklusion}
Målet med projektet var, at udvikle en DC/DC converter som skal kunne indgå i et universelt aktiveringskredsløb. Her skulle det være muligt at tilpasse converteren til to forskellige load typer. 

Der er blevet implementeret en funktionsdygtig converter med en statisk udgang. Desuden opfylder converteren de fleste krav for den valgte udgangsbelastning. Samtidig er der blevet lagt et grundlag, og gjort nogle overvejelser, for videreudviklingen af converterens udgangstrin. 

Der er blevet udviklet en converter med hurtig og stabil regulering. Reguleringen overholder kravene for gain- og fasemargin for den valgte belastning, inden for hele indgangsspænding intervallet. Desuden overholder den præcisionskravene for både udgangsstrøm og -spænding, ved den valgte belastning. 

Der er blevet gjort overvejelser ift. et optimalt termisk design. Der er løbende i projektet blevet optimeret på dette punkt, men det endelige resultat er ikke tilfredsstillende. Desuden vil dette tab blive større hvis udgangsbelastningen øges. Derfor er der blevet gjort nogle overvejelser, for hvordan dette krav vil blive overholdt. 

Der er blevet opstillet en funktionel P-spice model, der giver et ret præcist indblik i converterens funktionalitet. Modellen er så tilfredsstillende, at stort set samtlige funktionaliteter kan efter vises. Der er dog mindre afvigelser da modellen for den ønskede MOSFET ikke kunne skaffes.

