
\chapter{Konklusion}
Målet med projektet var, at udvikle en DC/DC converter, som skal kunne indgå i et universelt aktiveringskredsløb. Her skulle det være muligt, at tilpasse converteren til to forskellige belastningstyper. 

Der er blevet implementeret en funktionsdygtig converter, med en statisk udgang. Desuden opfylder converteren de fleste krav for den valgte udgangsbelastning. Samtidig er der blevet lagt et grundlag, og gjort nogle overvejelser, for videreudviklingen af converterens udgangstrin. 

Der er udviklet en converter med hurtig og stabil regulering. Reguleringen overholder kravene til gain- og fasemargin for den valgte belastning, inden for hele indgangsspændings-intervallet. Desuden overholder den præcisionskravene for både udgangsstrøm og -spænding, ved den valgte belastning. 

Der er blevet gjort overvejelser ift. et optimalt termisk design. Der er løbende i projektet blevet optimeret på dette punkt, men det endelige resultat er ikke tilfredsstillende. Desuden vil dette tab blive større hvis udgangsbelastningen øges. Derfor er der blevet gjort nogle overvejelser for, hvordan kravet vil blive overholdt. 

Der er opstillet en funktionel P-spice model, der giver et præcist indblik i converterens funktionalitet. Modellen er så tilfredsstillende, at stort set samtlige funktionaliteter kan eftervises. Der er dog mindre afvigelser, da modellen for den ønskede MOSFET ikke kunne skaffes.

