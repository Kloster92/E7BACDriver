\chapter{Kravspecifikation}

%\subsection{Prioritering / Afgrænsning} måske?
%Moscow
Kravene til produktet er prioriteret ved brug af MoSCoW metoden. Her er kravene for produktet inddelt i fire kategorier, hvor de vigtigste elementer er prioriteret højest. \textbf{Must} benævner de krav som er vigtigst at opfylde, og som er absolut nødvendigt for produktet. \textbf{Should} er de krav produktet bør opfylde. \textbf{Could} er kravene som produktet evt. kunne opfylde, hvis projektets tidsramme tillader det. \textbf{Won't} er krav som ikke vil blive opfyldt inden for projektets tidsrammer, men evt. kan tages med i senere iterationer.

\noindent Følgende opdeling viser kravene udvalgt for dette projekt:
\begin{itemize}
	\item[\textbf{Must}]
		\begin{itemize}
			\item Holde konstant udgangsstrøm og -spænding
			\item Have stabil regulering
			\item Ikke påvirke andre moduler ved fejl
			\item Konstrueres med EEE komponenter

		\end{itemize}
	\item[\textbf{Should}]
		\begin{itemize}
			\item Have programmerbar udgangsstrøm og -spænding

		\end{itemize}
	\item[\textbf{Could}] 
		\begin{itemize}
			\item Have overstrømsbeskyttelse på udgangen

		\end{itemize}
	\item[\textbf{Won't}]
		\begin{itemize}
			\item Indeholde galvanisk adskillelse
		\end{itemize}
\end{itemize}
% De krav der er kritisk for selve funktionen af bilen, dvs. bevægelse / styring, er prioriteret højest. Herefter er krav, som afhænger af bla bla bla. Måske.

%\begin{figure}[H]
%	\centering
%	\includegraphics{tex/Kravspecifikation/Billeder/Aktor-kontekst_diagram_v0_1.pdf}
%	\caption{Aktør-kontekst diagram}
%\end{figure}

%\begin{figure}[H]
%	\centering
%	\includegraphics{tex/Kravspecifikation/Billeder/Usecase_diagram_v1_0.pdf}
%	\caption{Use case diagram}
%\end{figure}

\section{Aktørbeskrivelse}
I det følgende afsnit beskrives systemets aktører. Ved hver aktør angives typen, samt en kort beskrivelse af aktørens funktion og/eller hvordan de påvirker systemet.

\begin{framed}
	\subsection{Aktør: Bruger}
	\subsubsection*{Type:}
		Primær
	
	\subsubsection*{Beskrivelse:}
		Brugeren interagerer med systemet via et interface.
		
		Han kan indstille den ønskede fart, samt kontrollere den nuværende fart og \indent spændingen på batterierne
\end{framed}

\clearpage


\section{Fully dressed use cases}

\begin{framed}
	\subsection{Use case 1 - Start bil}
	\subsubsection*{Mål:}
		Initiere bilen så den er klar til kørsel og er klar til at modtage input
		
	\subsubsection*{Initiering:}
		Brugeren
	
	\subsubsection*{Aktører:}
		Brugeren (primær)
	
	\subsubsection*{Referencer:}
		Ingen
	
	\subsubsection*{Samtidige forekomster:}
		En
	
	\subsubsection*{Forudsætning:}
		Bilen er slukket og der er forbindelse fra interface til bil
	
	\subsubsection*{Resultat:}
		Bilens sensorer er tændt, motorer er klar, bilen holder stille
	
	\subsubsection*{Hovedscenarie:}
		\begin{enumerate}
			\item Brugeren vælger via interface ''Start bil''
			\item Bilen monitorerer sensorinputs og rapporterer status 
			\item Bilen udfører motortjek ved at køre bilen lidt frem og derefter tilbage
			\item Bilen rapporterer status
			\item Bilen tænder for- og baglys, blinker med blinklys hvis status er OK 
			\begin{description}
					\item[Extension 1:] Status ikke OK
			\end{description}
			\item Bilen afventer brugerinput
		\end{enumerate}
	
	\subsubsection*{Extensions:}
	\textbf{Extension 1:} Status ikke OK	% Fix layout
		\begin{enumerate}
			\item Bilen rapporterer fejl og forsøger at angive hvilken sensor og/eller motor der fejler
		\end{enumerate}
	
\end{framed}
	

% Skabelon
%\begin{framed}
%\subsubsection{Mål:}
%
%\subsubsection{Initiering:}
%
%\subsubsection{Aktører:}
%
%\subsubsection{Referencer:}
%
%\subsubsection{Samtidige forekomster:}
%
%\subsubsection{Forudsætning:}
%
%\subsubsection{Resultat:}
%
%\subsubsection{Hovedscenarie:}
%
%\subsubsection{Extension:}

\section{Ikke-funktionelle krav}
I dette afsnit beskrives de ikke-funktionelle krav. Her opstilles f.eks. krav om præcision, brugervenlighed samt produktets dimensioner.
\begin{itemize}
			\item Bilens længde og bredde må ikke overskride 50cm x 30 cm
			\item Brugeren skal have mulighed for at kommunikere med bilen via tekst-terminal og / eller grafisk brugergrænseflade
			\item Bilen skal detektere objekter indenfor intervallet 20cm til 2,5m
			\item Bilen skal kunne køre med en fart på 13 km/t$\pm$ 0,5 
			\item Bilen skal have en maksimal vægt på 2kg
			\item Bilen skal måle sin fart i intervallet 0-13 km/t$\pm$ 0,5
			\item Bilen skal kunne indstille sin fart i intervallet 0-13 km/t med en opløsning på 0,5 km/t	
\end{itemize}
