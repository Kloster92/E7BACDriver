
\chapter{Projektafgrænsning}
I dette afsnit er der beskrevet en afgrænsning af projektets indhold. Her er der taget udgangspunkt i det den ønskede funktionalitet af produktet, som sammenholdes med den egentlige opnåede funktionalitet. Elementer af projektet der er specificeret, men ikke implementeret er angivet i dette afsnit og uddybet i afsnit~\ref{future}. 

Produktets kernefunktionaliteter er prioriteret under udviklingen. Her er især prioriteret efter elementer, som kræves funktionsdygtige, før andre elementer kan udvikles. Her er udviklet en base for produktet, hvorpå flere krav og funktionaliteter skal kunne påføres.

Der blev valgt, at tage udgangspunkt i en converter med en statisk udgang. Dette ville skabe et udgangspunkt, og genere en erfaring, der ville give base for en videreudvikling til en dynamisk udgang af converteren.  

Det er valgt, at vikling af transformatoren sker af gruppen selv. Dette vil give en erfaring inden for området, der giver indsigt i hvad der er realistisk at designe efter. Derudover vil det også give et indblik i problematikkerne i vikling af en transformator. 

Converteren er designet efter de termiske krav, ved løbende vurdering og optimering af effekttabet i komponenterne.

Ved udvikling af elektroniske produkter til rumfart, kræves det udviklet med EEE-komponenter. Disse er meget omkostningsfulde, og er derfor ikke blevet brugt i projektet. Til gengælde er der brugt Terma-godkendte komponenter. Med disse komponenter har Terma en erfaring med, at opsætte disse til EEE-komponenter.  

Det blev specificeret at converteren skulle kunne operere ved et specifikt temperaturinterval. Det blev sikret ved kontrol af komponenternes specifikationer.

For at sikre en stabil indgangsspænding på converteren er det nødvendigt at implementere et indgangsfilter. Dette filter er blevet stillet tilrådighed af Terma, da det blev besluttet at fokusere andetsteds. Dette filter vil ikke blive beskrevet i denne rapport, men funktionaliteten af det er beskrevet i dokumentationen.

For at sikre converteren kun vil være afhængig af indgangsspændingen til power-modulet, skal der udvikles en regulator til forsyning af PWM-controlleren. I projektet bliver denne forsyningsspænding dog realiseret ved en ekstern spændingskilde. 




