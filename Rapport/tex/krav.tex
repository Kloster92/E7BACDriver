\chapter{Krav}
% Præsenter krav osv.
Kravene til produktet er analyseret vha. MoSCoW-metoden\cite{MoSCoW}. MoSCoW-metoden inddeler kravene til et system i fire kategorier. De fire kategorier er: Must, Should, Could og Won't.

\begin{itemize}
	\item[\textbf{Must}]
	\begin{itemize}
		\item Køre fremad og dreje
		\item Holde en fastsat brugerdefineret fart
		\item Holde sig inden for vejstriber
		\item Rapportere hastighed til bruger
		\item Trådløs kommunikation med brugeren
	\end{itemize}
	\item[\textbf{Should}]
	\begin{itemize}
		\item Tilpasse fart efter forankørende
		\item Monitorere og rapportere batteri-niveau
		\item Overhale ved brugerkommando
		\item Stoppe ved forhindring på kørebanen
	\end{itemize}
	\item[\textbf{Could}] 
	\begin{itemize}
		\item Blinklys og kørelys
		\item Undvige forhindring
		\item Grafisk brugergrænseflade
		\item Køre baglæns
	\end{itemize}
	\item[\textbf{Won't}]
	\begin{itemize}
		\item Have navigation
		\item Læse vejskilte
		\item SmartPhone-applikation til styring
	\end{itemize}
\end{itemize}

Ud fra ovenstående MoSCoW-model er der fremstillet en række krav. Der er funktionelle krav, som defineres ved use cases, og ikke-funktionelle krav.

\section{Kravspecifikation}
Kravspecifikationen indeholder kravene til projektet. Use casene er funktionelle krav til systemet set fra brugerens perspektiv.  
I diagrammet nedenfor beskrives brugerens og systemets sammenhæng til de funktionelle krav for produktet. 
Brugeren kan via kommandoer på grænsefladen styre bilen, ved at initiere de forskellige use cases. De andre aktører ses til højre på figuren, og linjerne viser hvilke use cases de er involveret i. 
%\begin{figure}[H]
%	\centering
%	\includegraphics{tex/krav/Billeder/Usecase_diagram_v1_0.pdf}
%	\caption{Use case diagram}
%\end{figure}

\subsection{Use cases}
De seks use cases forklares i dette afsnit. For yderligere beskrivelse af de forskellige use cases, se dokumentationens kravspecifikation. 

\noindent\textbf{Use case 1: Start bil}
\\Use casen starter bilen når brugeren vælger ''Start'' via grænsefladen. Herefter vil bilen monitorere sensorinputs, udføre motortjek og rapportere status til brugeren via grænseflade.

\noindent\textbf{Use case 2: Rapportér driftstatus}
\\Use casen rapporterer om driftstatus når brugeren vælger ''Status'' via grænsefladen. Dette indebærer batteriniveau, bilens fart, status på vejbanesensorer og afstand til objekt foran. 

\noindent\textbf{Use case 3: Ændr fart}
\\Brugeren indstiller den ønskede fart via grænsefladen og use casen justerer derefter farten til den ønskede værdi.

\noindent\textbf{Use case 4: Følg vejbane}
\\Use casen sørger for at bilen korrigerer sin retning i forhold til vejbanen. Dette er den eneste use case, som ikke initieres af brugeren. 

\noindent\textbf{Use case 5: Overhal}
\\Use casen overhaler et forankørende objekt når brugeren anmoder om en overhaling via grænsefladen.

\noindent\textbf{Use case 6: Sluk bil}
\\Use casen slukker bilen når brugeren vælger ''Sluk'' via grænsefladen. Her vil bilens fart sættes til nul samt slukke for sensorer, lys og motor. Dette udskrives på grænsefladen.

\subsection{Ikke-funktionelle krav}
Kravene for kvaliteten af bilen beskrives ved ikke-funktionelle krav. Disse krav er beskrevet under kravspecifikationen i dokumentationen. 
\\Kravene indebærer bilens dimensioner i form af størrelse og vægt. Der stilles krav til bilens topfart og sonarens rækkevidde. 
Yderligere er der krav til, at brugeren skal kunne kommunikere med bilen via en grafisk brugergrænseflade.   
%% Krav:
% Aktørbeskrivelse...
% Præsenter hvordan systemet ser ud for brugeren / aktørernes perspektiv. Hvad er det for nogle aktører

% Use cases:
% Vis use-cases og fortæl hvad de overordnet handler om.

% er der mere af krav til projektet evt.?