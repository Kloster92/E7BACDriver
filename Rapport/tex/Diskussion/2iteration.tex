
\section{Anden iteration}
I 2. iteration blev den første version af converteren implementeret.

Der blev viklet en transformator med en selvinduktion på $57.7\micro H$, hvor der blev designet efter en induktion på $57.76\micro H$. Denne afvigelse vurderes ubetydelig, og derfor godtages dette. Yderligere blev spredningsselvinduktionen målt til $152nH$. Dette giver en kobling i transformatoren på $99.73\percent$. Denne kobling vurderes tilfredsstillende, og derfor vurderes transformatoren samlet som godkendt. 

På udgangen af converteren blev en DC-spænding på $20.97V$. Dette er indenfor kravet for en afvigelse på $21V \pm2\percent$, derfor godkendes dette. Der blev tilgengæld målt switching-spikes på udgangen på omkring $3.5V$ pk-pk. Dette opfylder ikke kravet på $100mV$ pk-pk, og derfor blev dette optimeret i 3. iteration. 

Den stationære drain-spænding over MOSFET'en blev målt til lidt over $47V$. Med en indgangsspænding på $26V$ og en udgangsspænding på $21V$, var dette som analyseret. Desuden blev der målt peak-spændinger op mod $80V$. For at tage højde for disse spikes blev der valgt en MOSFET med en breakdown spænding på $150V$. Samme måling blev foretaget for dioden. Her blev der målt en stationær spænding over den på $45V$, med spikes op mod $67V$. Her blev der valgt en diode med en breakdown spænding på $120V$. Da der ved begge komponenter er en stor margin til de respektive breakdown spændinger godtages disse komponentvalg.

Der blev dog målt observeret, at der blev anslået svingninger på spændingerne over MOSFET og diode. Derfor blev det valgt at udvikle snubber-kredsløb, til dæmpning af disse svingninger. 

 