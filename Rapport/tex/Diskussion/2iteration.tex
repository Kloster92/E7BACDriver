
\section{Anden iteration}
I 2. iteration blev den første version af converteren implementeret.

\noindent Der blev viklet en transformator med en selvinduktion på $57.7\micro H$, hvor der blev designet efter en induktion på $57.76\micro H$. Denne afvigelse vurderes ubetydelig, og derfor godtages dette. Yderligere blev spredningsselvinduktionen målt til $152nH$. Det giver en kobling i transformatoren på $99.73\percent$. Koblingen vurderes tilfredsstillende, og derfor vurderes transformatoren samlet som godkendt (Dokumentationen afsnit 5.1.5, figur: 5.12 og 5.13). 

På udgangen af converteren blev en DC-spænding på $20.97V$ målt. Det er indenfor kravet for en afvigelse på $21V \pm2\percent$, derfor godkendes dette. Der blev tilgengæld målt switching-spikes på udgangen, på omkring $3.5V$ pk-pk. Det opfylder ikke kravet på $100mV$ pk-pk, og derfor blev dette optimeret i 3. iteration. 

Den stationære drain-spænding over MOSFET'en blev målt til lidt over $47V$. Med en indgangsspænding på $26V$ og en udgangsspænding på $21V$, var dette som analyseret. Desuden blev der målt peak-spændinger op mod $80V$. For at tage højde for disse spikes blev der valgt en MOSFET, med en breakdown spænding på $150V$. Samme måling blev foretaget for dioden. Her blev der målt en stationær spænding over den på $45V$, med spikes op mod $67V$. Her blev der valgt en diode med en breakdown spænding på $120V$ (Dokumentationen, afsnit 5.9.1, figur: 5.64). Da der ved begge komponenter er en stor margin, til de respektive breakdown spændinger, godtages disse komponentvalg.

Der blev dog observeret, at der blev anslået svingninger på spændingerne over MOSFET og diode (Dokumentationen, afsnit 5.9.1, figur: 5.65). Derfor blev det valgt, at udvikle snubber-kredsløb, til dæmpning af disse svingninger. 

Switch-frekvensen i PWM-controlleren blev målt til $95.8k\hertz$, og blev designet efter $100k\hertz$. Ved tolerancer i komponenter og controller, godtages denne afvigelse. Switch-tiden i MOSFET'en blev målt til $120ns$, og blev designet efter $138.7ns$. Stigetiden i current-sense filteret blev målt til $350ns$, som blev designet efter $300ns$. Switch-tiden og stigetiden blev designet langsomt, således der blev sikret en stabil funktionalitet, og derfor blev det valgt, at optimere disse i 3. iteration (Dokumentationen, afsnit 5.9.2). 

Båndbredden i controlleren blev målt til $900\hertz$, ved en gain-margin på $24\decibel$ og en fasemargin på $74.3^\circ$. Båndbredden blev designet lav, for at sikre en stabil funktionalitet. Der blev opnået en stor margin til reguleringskravene, derfor blev det valgt, at optimere båndbredden til 3. iteration. 

Ved et load-step på converterens udgang, blev der målt et udfald på udgangsspændingen på op til $700mV$, før den blev reguleret tilbage til udgangspunktet. Formålet med optimeringen af båndbredden, var også at mindske dette udfald. Desuden blev tiden før spændingen var indreguleret igen, målt til $1.5ms$. 

Det samlede tab i converteren blev målt til $8.6W$. Det var større end kravet på et maksimalt effekttab på $5W$. Det samlede MOSFET tab blev målt til $5W$, og derfor blev det valgt at optimere dette tab i 3. iteration (Dokumentationen, afsnit 5.9.5, tabel 5.11). 








