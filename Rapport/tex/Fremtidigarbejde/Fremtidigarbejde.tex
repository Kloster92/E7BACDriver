\chapter{Fremtidig arbejde} \label{future}
Dette afsnit beskriver, hvordan det er tænkt, at løse de krav fra MoSCoW'en, som endnu ikke er implementeret. Rækkefølgen punkterne i det fremtidige arbejde forventes, at blive udført, sker efter MoSCoW'en. Først vil kravene fra \textit{should} blive færdiggjort og herefter kravene i \textit{could}.  

Der er lavet overstrømsbeskyttelse i converteren. Det skal dog vurderes, om beskyttelsen er god nok, eller om strømmen kan blive for høj inden beskyttelsen indtræffer. Det kommer an på hvilken strøm, de resterende komponenter kan holde til. Det vedrører også kravet om, ikke at påvirke andre moduler ved fejl. Bliver strømmen høj før overstrømsbeskyttelsen tager over, kan andre moduler blive påvirket af dette. Det krav er der til dels taget højde for. Designet af converteren bruger udelukkende komponenter, der kan håndtere, at blive sendt ud i rummet. Det betyder foreksempel, at elektrolytter har været udelukket, da disse vil eksplodere under rumfart. Det ville udover at ødelægge converteren, samtidig påvirke de resterende moduler.

I 4. iteration er det meningen, at kravet til en ekstra belastningstype skal implementeres. Måden det er tænkt at gøre det på, er at tilføre et ekstra reguleringsloop til kredsløbet. Det vil blive forsøgt implementeret med en ekstra current-sense modstand på sekundærsiden, til at måle strømmen her. Strømmålingen skal kobles ind, som en del af det eksisterende feedback loop.

Om det termiske design er kompatibelt med vakuum, er endnu ikke testet. Det sker, når det samlede tab er reduceret til maksimalt $5W$. Her vil converteren blive testet i et vakuumrum. For at reducere tabet yderligere, end det er tilfældet i 3. iteration, er der tænkt i flere optimeringspunkter. Det er muligt, at finde en MOSFET og diode, med mindre ON modstand og spændingsfald, for at optimere disse tab til det yderste. 
Fra overblikket over det samlede tab efter 3. iteration, kan det ses, at tabet i current-sense modstanden, fylder en stor del af det samlede tab. Tabet vil stort set kunne fjernes, hvis der implementeres en strøm-transformator i stedet. Her vil strømmen, som løber i modstanden på nuværende tidspunkt, kunne transformeres ned, med et omsætningsforhold. Ved at indsætte en current-sense modstand på sekundærviklingen af strømtransformatoren, vil tabet kunne mindskes med omsætningsforholdet.  

Udover det ovenstående er der to punkter i accepttesten, som endnu ikke er udført. Det indebærer temperaturintervallet, som converteren skal operere indenfor, samt converterens samlede dimensioner. Begge dele er krav, der i løbet af designet er taget højde for. Da det ikke er et færdigt layout, men en prototype, kan testenene endnu ikke foretages. Med hensyn til temperaturen, er der brugt komponenter, som i følge databladene bør overholde temperaturkravene. Ved komponenternes størrelser, er det sikret, at finde komponenter, der ikke er højere end det tilladte.    

Indtil nu har PWM-controlleren været forsynet af en ekstern 12V spænding. Inden printet skal lægges endeligt ud, skal dette laves om. Der vil blive lavet en preregulator, som sikrer, at PWM-controlleren forsynes. Når converteren er oppe at køre, vil PWM-controlleren få sin forsyning fra udgangen. Det kan ikke lade sig gøre under opstart. Derfor skal det implementeres sådan, at controlleren i begyndelsen får sin VCC fra indgangsspændingen, og herefter fra udgangen. 

