\chapter{Fremtidig arbejde}
Dette afsnit beskriver hvordan det er tænkt at løse de krav fra MoSCoW'en, som endnu ikke er implementeret. Rækkefølgen punkterne i det fremtidige arbejde forventes at blive udført, sker efter MoSCoW'en. Først vil kravene fra should blive færdiggjort og herefter kravene i could.  

Der er lavet overstrømsbeskyttelse i converteren. Det skal dog vurderes om beskyttelsen er god nok, eller om strømmen kan blive for høj inden beskyttelsen indtræffer. Det kommer an på hvilken strøm de resterende komponenter kan holde til. Det vedrører også kravet om ikke at påvirke andre moduler ved fejl. Bliver strømmen høj før overstrømsbeskyttelsen tager over, kan andre moduler blive påvirket af dette. Dette krav er der dog til dels taget højde for. Designet af converteren bruger udelukkende komponenter, der kan håndtere at blive sendt ud i rummet. Det betyder foreksempel at elektrolytter har været udelukket, da disse vil eksplodere under rumfart. Dette ville udover at ødelægge converteren samtidig påvirke de resterende moduler.

I 4. iteration er det meningen at kravet til en ekstra load type skal implementeres. Måden det er tænkt at gøre dette på, er at tilføre et ekstra reguleringsloop til kredsløbet. Det vil blive forsøgt implementeret med en ekstra current sense modstand på sekundærsidenm til at måle strømmen her. Strømmen her skal så kobles ind, som en del af feedback loopet.

Om det termiske design er kompatibelt med vakuum er endnu ikke testet. Dette sker, når det samlede tabet er reduceret til de 5W. Her vil converteren blive testet i et rum i vakuum. For at reducere tabet yderligere end det er tilfældet i 3. iteration er der tænkt i flere optimeringspunkter. Det er muligt at finde en MOSFET og diode med mindre ON modstand og spændingsfald, for at optimere disse tab til det yderste. 
Fra overblikket over det samlede tab efter 3. iteration, er det dog current sense modstanden, der har det højeste tab. Dette tab vil stort set kunne fjernes, hvis der implementeres en strøm-transformator. Her vil strømmen, som løber i modstanden på nuværende tidspunkt kunne transformeres ned med et omsætningsforhold. Ved at indsætte sin current-sense modstand på den vikling af transformatoren med en transformeret strøm, vil tabet kunne mindskes betydeligt.  

Udover det ovenstående er der to punkter i accepttesten, som endnu ikke er udført. Det indebærer temperaturintervallet, converteren skal operere indenfor, samt converterens samlede dimensioner. Begge dele er krav, der i løbet af designet er taget højde for. Da det ikke er et færdigt layout, men en prototype, kan testenene endnu ikke foretages. Med hensyn til temperaturen, er der brugt komponenter, som i følge databladene bør overholde temperaturkravene. Ved komponenternes størrelser, er det sikret, at finde komponenter der, ikke er højere end det tilladte.    

